\documentclass[ %handout, % for handouts %%% 12pt,handout,
 10pt, xcolor={dvipsnames,svgnames,x11names,hyperref},
   hyperref={colorlinks=true,citecolor=green,linkcolor=DarkRed,urlcolor=ProcessBlue,anchorcolor=blue}
  ]{beamer}
\usepackage[ruled, linesnumbered]{algorithm2e}

\usepackage{tcolorbox}
 \tcbuselibrary{skins,raster}
\usepackage{outlines}
\usepackage{multirow}
\usepackage{babel}
\usepackage{blindtext}
\usepackage{verbatim}
\usepackage{minted}
\usepackage{tabularx}


\usepackage{pgfplots}
\pgfplotsset{compat=1.14}
\usepgfplotslibrary{statistics}

\setbeamertemplate{navigation symbols}{}

 \usepackage{relsize}

\usepackage{bookmark}

%\usepackage[hyperref]{xcolor}


\let\oldcite=\cite
\renewcommand{\cite}[1]{\textcolor[rgb]{.7,.7,.7}{\oldcite{#1}}}


\mode<presentation> {

% The Beamer class comes with a number of default slide themes
% which change the colors and layouts of slides. Below this is a list
% of all the themes, uncomment each in turn to see what they look like.

%\usetheme{default}
%\usetheme{AnnArbor}
%\usetheme{Antibes}
%\usetheme{Bergen}
%\usetheme{Berkeley}
%\usetheme{Berlin}
%\usetheme{Boadilla}
%\usetheme{CambridgeUS}
%\usetheme{Copenhagen}
%\usetheme{Darmstadt}
%\usetheme{Dresden}
%\usetheme{Frankfurt}
%\usetheme{Goettingen}
%\usetheme{Hannover}
%\usetheme{Ilmenau}
%\usetheme{JuanLesPins}
%\usetheme{Luebeck}
\usetheme{Madrid}
%\usetheme{Malmoe}
%\usetheme{Marburg}
%\usetheme{Montpellier}
%\usetheme{PaloAlto}
%\usetheme{Pittsburgh}
%\usetheme{Rochester}
%\usetheme{Singapore}
%\usetheme{Szeged}
%\usetheme{Warsaw}

% As well as themes, the Beamer class has a number of color themes
% for any slide theme. Uncomment each of these in turn to see how it
% changes the colors of your current slide theme.

%\usecolortheme{albatross}
%\usecolortheme{beaver}
%\usecolortheme{beetle}
%\usecolortheme{crane}
\usecolortheme{dolphin}
%\usecolortheme{dove}
%\usecolortheme{fly}
%\usecolortheme{lily}
%\usecolortheme{orchid}
%\usecolortheme{rose}
%\usecolortheme{seagull}
%\usecolortheme{seahorse}
%\usecolortheme{whale}
%\usecolortheme{wolverine}

%\setbeamertemplate{footline} % To remove the footer line in all slides uncomment this line
%\setbeamertemplate{footline}[page number] % To replace the footer line in all slides with a simple slide count uncomment this line

%\setbeamertemplate{navigation symbols}{} % To remove the navigation symbols from the bottom of all slides uncomment this line
}

\setbeamercolor{alerted text}{fg=red}


\usepackage[absolute,overlay]{textpos}
\usepackage{graphicx}
\usepackage{booktabs} % Allows the use of \toprule, \midrule and \bottomrule in tables
\usepackage{forest}
 \usepackage{tikz}
 \usetikzlibrary{shapes.geometric}
\usepackage{rotating}
\usepackage[]{wrapfig}
\usetikzlibrary{arrows,shapes}
\usetikzlibrary{trees,matrix}
\usepackage{multirow}
\usepackage{dirtree}
%\usepackage{color, colortbl}
\definecolor{Gray}{gray}{0.85}
\newcommand\x{.11}
\graphicspath{ {Images/} }
\usepackage{mathrsfs}
%\usepackage[symbol]{footmisc}

%\usepackage{xcolor}
%\hypersetup{
%  colorlinks,
%  allcolors=.,
%  urlcolor=ProcessBlue,
%}
%\hypersetup{colorlinks = true,
%%            linkcolor = red,
 %           urlcolor=ProcessBlue,
 %           citecolor = green,
 %           anchorcolor = blue}

%\usepackage{bibunits}
%\setbeamertemplate{bibliography item}{[\theenumiv]}
%\defaultbibliography{IoT,CPS}
%\defaultbibliographystyle{IEEEtran}

\usepackage[numbers]{natbib}
\usepackage{bibunits}

%If by "animated" you mean creating overlays, then a straight application of Daniel's visible on key would solve the problem.
%Step 1. Put the following in the preamble:
\tikzset{
    invisible/.style={opacity=0,text opacity=0},
    visible on/.style={alt=#1{}{invisible}},
    alt/.code args={<#1>#2#3}{%
      \alt<#1>{\pgfkeysalso{#2}}{\pgfkeysalso{#3}} % \pgfkeysalso doesn't change the path
    },
}
\forestset{
  visible on/.style={
    %for children={
      /tikz/visible on={#1},
      edge={/tikz/visible on={#1}}
   % }
  }
}


\AtBeginSection[]
{
  \begin{frame}<beamer>
    \frametitle{Outline}
    \tableofcontents[currentsection]
  \end{frame}
}

\AtBeginSubsection[]
{
  \begin{frame}<beamer>
    \frametitle{Outline}
    \tableofcontents[currentsubsection]
  \end{frame}
}

% \logo{\raisebox{-0.5cm}{\includegraphics[width=1cm]{naulogo.png}}\hspace*{0.5cm}}

\newenvironment{stepitemize}{\begin{itemize}[<+->]}{\end{itemize} }

\newcommand{\myblue}{\only{\color{blue}}}
\newcommand{\mygreen}{\only{\color{green}}}
\newcommand{\myyellow}{\only{\color{yellow}}}
\newcommand{\myorange}{\only{\color{orange}}}
\newcommand{\myred}{\only{\color{red}}}
\newcommand{\Z}{\mathbb{Z}}
\newcommand{\Q}{\mathbb{Q}}
\newcommand{\R}{\mathbb{R}}
\newcommand{\C}{\mathbb{C}}
\newcommand{\RR}{\mbox{\msbm R}}
\newcommand{\F}{\mathbb{F}}
\newcommand{\Rs}{\mathcal{R}}
\newcommand{\Hs}{\mathbb{H}}

\begin{document}

\title[Rings and Fields Fundamentals]{
Rings and Fields Fundamentals for Homomorphic Encryption}
\author{Bahattin Yildiz}
\institute[DBIO-Glade]{DBIO-Glade}
\date[September]{September 2022}
\maketitle

\section{Introduction}

\begin{frame}{Why Rings and Fields?}
\begin{stepitemize}
    \item Addition and multiplication are the necessary operations in FHE.
    \item Groups have only one operation
    \item Rings and fields are equipped with the two operations
    \item Many well-known sets that appear in the HE are rings
    \item $\Z$, $\Q$, $\R$, $\C$, $\Z_q$, $\Z_p$, $\Z[x]$, $\Z_p[x]$, $\Z[x]/(f(x))$, $\Z_p[x]/(f(x))$ are all rings.
\end{stepitemize}
\end{frame}

\begin{frame}{Ring Definition}
\begin{stepitemize}
\item Let us start with the definition of a ring:
\begin{definition}
A nonempty set $R$ that is equipped with two binary operations (genericaly called ``addition" and ``multiplication") $+$ and $\cdot$ is called a {\bf ring} if it satisfies the following conditions:
\begin{enumerate}
    \item $(R,+)$ is an Abelian group
 \item $R$ is closed under multiplication, that is $a\cdot b \in R$ whenever $a,b\in R$
 \item Multiplication in $R$ is associative, that is $(ab)c=a(bc)$.
 \item The multiplication operation is distributive over addition. More precisely, $a(b+c)=ab+ac$ whenever $a, b, c \in R$.
\end{enumerate}
\end{definition}
\end{stepitemize}

\end{frame}

\begin{frame}{Some remarks}
    \begin{stepitemize}
    \item Since $(R,+)$ is an Abelian group, it has an identity, which we will denote by $0_R$ or just simply by $0$.
    \item A ring need not have a multiplicative identity. If there is a multiplicative identity, the ring is called a ``ring with identity".
    \item The identity is denoted by $1_R$ or just $1$.
    \item The multiplication does not have to be commutative: \item i.e., we do not need to have $ab=ba$ for all $a,b \in R$.
    \item If this condition holds for all elements in $R$, then the ring is called a ``commutative ring".
    \item We will mostly deal with ``nice" rings, that is we will mostly be dealing with commutative rings with identity.
    \end{stepitemize}
\end{frame}

\begin{frame}{Examples}
\begin{stepitemize}
\item $\Z$ is a commutative ring with identity with respect to the usual addition and multiplication operations.
    \item Similarly $\Q$, $\R$ and $\C$ are commutative rings with identity.
    \item For $n>1$, the set $\Z_n=\{0,1, \dots, n-1\}$ is a commutative ring with idnetity with respect to addition and multiplication modulo $n$.
    \item $2\Z$, the set of even integers is a commutative ring without identity.
    \item $\R^{n\times n}$, the set of $n\times n$ matrices over the real numbers is a non-commutative ring with identity. The additive identity in this case is the $0$ matrix whereas the multiplicative identity is $I_n$, the identity matrix.
    \item Let $\R[x]$ be the set of polynomials over $\R$. Then $\R[x]$ is a commutative ring with identity with respect to polynomial addition and polynomial multiplication. The multiplicative identity is the constant polynomial $1$, whereas the additive identity is just the zero polynomial.
    \item The set of polynomials of degree $\leq n$ is not a ring, while it is an Abelian group under addition. The reason is because this set is not closed under polynomial multiplication.
    \item Similar to the previous case, we can talk about
    $\Z[x]$ and $\Z_m[x]$ are also commutative rings with identity.
\end{stepitemize}

\end{frame}
\section{Zero Divisors and Units}
\begin{frame}{Zero Divisors and Units}
    \begin{stepitemize}
    \item A ring can have zero divisors, i.e., elements $a,b \neq 0$ such that $ab=0$.
    \item For example in $\Z_6$, we have $2\cdot 3=0$ while both $2$ and $3$ are non-zero in the ring.
    \item However we cannot find such elements in $\Z$ or in $\Z_7$.
    \item  An element $a\neq 0$ is said to be a ``zero divisor" if $ab=0$ for some $b\neq 0$ in $R$.
    \item
    A commutative ring $R$ with identity is called an ``integral domain" if it has no zero divisors.
    \item Equivalently $R$ is an integral domain if $ab=0$ in $R$ leads to $a=0$ or $b=0$.
    \item {\bf (Cancellation Property)} If $R$ is an integral domain, then $ab=ac$ implies $a=0$ or $b=c$.
    \item The cancellation would not work in a non-integral domain as for example in $\Z_6$ we have $2\cdot 2=2\cdot 5$, but $2\neq 5$ in $\Z_6$.
    \end{stepitemize}
\end{frame}

\begin{frame}{Examples}
\begin{stepitemize}
\item $\Z$, $\Q$, $\R$, $\C$ and $\Z_p$:$p$ prime
 are all examples of integral domains.
 \item If $n$ is composite then $\Z_n$ is not an integral domain because then by definition we can find $1<a,b<n$ such that $a\cdot b=n$ so we would have $ab=0$ in $\Z_n$, whereas $a, b\neq 0$.
\end{stepitemize}

\end{frame}
\begin{frame}{Units}
\begin{stepitemize}
\item An element $a\in R$ is said to be a ``unit" if there exits $b\in R$ such that $ab=1$.
\item So,  just like a zero divisor is a divisor of $0$, a unit is a divisor of $1$.
\item In $\Z$, $1$ and $-1$ are the only units.
    \item In $\Z_6$, $5$ is a unit since $5\cdot 5 =1$ in $\Z_6$. $2$ is not a unit since $2a$ will always be $0,2$ or $4$ modulo $6$.
    \item Recalling the Number Theory part, we see that $a\in \Z_n$ is a unit if and only $GCD(a,b)=1$.
    \item Thus in $\Z_p$, where $p$ is prime, every non-zero element is a unit.
    \item In $\Q$, $\R$ and in $\C$, every non-zero element is a unit.
\end{stepitemize}
    \end{frame}
\begin{frame}{}
    \begin{stepitemize}
    \item Just like the case of zero divisors, we have a special classification for rings based on the concept if units:
\item A commutative ring $R$ with identity is called a ``field" if every non-zero element is a unit.
\item
$\R, \Q, \C$, $\Z_p:p$ prime are all fields while $\Z$, and $\Z_n:n$ composite are not fields.
\item When $p$ is prime, the field $\Z_p$ is usually called a ``prime field". It is a basic example of a ``finite field". \item There are many extensions of $\Z_p$, all called finite fields that have sizes $p^m$, where $p$ is a prime.

\item Since units cannot be zero divisors any field is automatically an integral domain but the converse is not true in general.
\item For example, $\Z$ is an integral domain that is not a field.
\item However, in the case of finite rings, the two concepts coincide, that is any finite integral domain is a field.
    \end{stepitemize}
\end{frame}

\section{Subrings and Ideals}
\begin{frame}{Subrings and Ideals}
\begin{stepitemize}
\item Let $R$ be a commutative ring with identity, and $S\subseteq R$ be a nonempty subset.
\item $S$ is called a ``subring" of $R$ if it is a ring under the same operations.
\item Equivalently, $S\subseteq R$ is a subring if
$$a-b\in S, \:\:\: ab\in S, \:\:\:\:\:\:\forall a,b\in S.$$
\item An ideal is a more restrictive concept although similar to subrings.
\item A subset $I$ of $R$ is called an ``ideal" if
\begin{enumerate}
    \item $a-b\in I$ for all $a,b \in I$, and
    \item $ar \in I$ for all $a \in I$ and $r \in R$.
\end{enumerate}
\end{stepitemize}
\end{frame}

\begin{frame}{Examples}
    \begin{stepitemize}
    \item $\{0\}$ and $R$ are ideals, which are labeled as ``trivial" ideals.
    \item $m\Z$ is an deal of $\Z$.
    \item $\Z$ is a subring of $\Q$, but it is not an ideal of $\Q$ since $2\cdot \frac{1}{3} \not \in \Z$.
    \item The set $I = \{p(x) \in \R[x]|p(0)=0\}$ is an ideal of $\R[x]$.
    \item This is basically the set of polynomials with zero constant terms.
    \item However the set $\{p(x)\in \R[x]|p(0)=1\}$ is not an ideal, nor in fact is it a subring. Since $p(0)-q(0)=0$ and not $1$.
    \item The set $I = \{p(x) \in \Z[x] | p(0) \:\:\textrm{is} \:\:\textrm{even}\}$ is an ideal of $\Z[x]$.
    \item More generally the set of polynomials whose constant terms are divisible by a prime $p$ is also an ideal.
    \item $\{0,2,4,6\}$ is an ideal of $\Z_8$.
    \end{stepitemize}
\end{frame}

\begin{frame}{Ideals of Fields}
    \begin{stepitemize}
    \item If an ideal $I$ contains $1$ then it must contain $1\cdot r$ for all $r\in R$ and hence we must have $I=R$.
    \item This is also true if $I$ contains a unit (since a unit times an element of $R$ is 1).
    \item This tells us that a field does not have any non-trivial ideals.
    \item In other words $\{0\}$ and $R$ are the only ideals of a field $R$.
    \end{stepitemize}
\end{frame}
\begin{frame}{Principal Ideals}
\begin{stepitemize}
\item For an element $a$ in a commutative ring with identity, we let
$(a) = \{ar|r\in R\}$.
\item Clearly $(a)$ is an ideal, which is called a ``principal ideal".
\item In some rings all ideals are principal.
\item $\Z, \Z_n$ are examples of such rings.
\end{stepitemize}
\end{frame}

\section{Ring Homomorphisms}
\begin{frame}{Ring Homomorphisms}
    \begin{stepitemize}
    \item A ring homomorphism is a map between rings that preserves the ring structure.
    \item More precisely we have:
\begin{definition}
Let $R$ and $R'$ be two rings. A function $\varphi:R\rightarrow R'$ is called a (ring) homomorphism if
$$\varphi(g_1+g_2) = \varphi(g_1)+\varphi(g_2), \:\:\:\:\textrm{and}\:\:\: \varphi(g_1g_2) = \varphi(g_1)\varphi(g_2) \:\:\: \forall g_1, g_2 \in G.$$
\end{definition}
\end{stepitemize}
\end{frame}


\begin{frame}{Examples}
\begin{stepitemize}
\item For any ring $R$, the map $\varphi:R \rightarrow R$ given by $\varphi(r)=0$ as well as the map $\phi:R \rightarrow R$ given by $\phi(r)=r$ are ring homomorphisms. These are also called ``trivial homomorphisms".
    \item It is not difficult to show that there is no non-trivial ring homomorphism from $\Z$ to $\Z$.
    \item For example $\phi(m)=2m$ is not a ring homomorphism since it is additive but not multiplicative.
    \item Similarly $\phi(m)=m^2$ is also not a ring homomorphism since it is multiplicative but not additive.
    \item $\phi:\Z \rightarrow \Z_n$ given by $\varphi(m)=(m)_n$ is a ring homomorphism since modulo reduction is both additive and multiplicative.
    \item $\phi:\R[x]\rightarrow \R$ given by $\phi(p(x)) = p(\alpha)$ for some $\alpha \in \R$ is a ring homomorphism also called the ``evaluation homomorphism".
   \item The RSA encryption is essentially a map from $\Z_n$ to $\Z_n$ that takes an element(message) $m$ to $m^e$ modulo $n$.
   \item This is clearly a multiplicative function but it is not additive, which means it is not a ring homomorphism.
\end{stepitemize}
\end{frame}

\begin{frame}{Kernel and Range}
\begin{stepitemize}
\item Just as in the case of group homomorphisms, we can talk about the kernel and the range of a homomorphism:
\item Let $\varphi:R\rightarrow R'$ be a ring homomorpshim. \item $ker(\varphi) = \{r\in R|\varphi(r)=0\}.$
\item $ran(\varphi) = \{\varphi(r)|r\in R\}.$
\item The kernel of a ring homomorphism is an ideal of $R$, whereas the range is a subring of $R'$.
\item For example, for $\phi:\Z \rightarrow \Z_n$ given by $\varphi(m)=(m)_n$, we have $ker(\varphi) = n\Z = (n)$, while $ran(\varphi) = \Z_n$.
\item For $\varphi:\R[x]\rightarrow \R$ given by $\varphi(p(x))=p(0)$, the kernel is all polynomialis with $0$ constant term, while the range is all of $\R$.
\end{stepitemize}
\end{frame}

\section{Ring Isomorphism}
\begin{frame}{Ring Isomorphism}
    \begin{stepitemize}
    \item A ring homomorphism $\varphi:R\rightarrow S$ is called an ``isomorphism" if it is one-to-one and onto.
    \item In the language of kernel and range we can say $\varphi$ is an ispomorphism if and only if $ker(\varphi)=\{0\}$ and $ran(\varphi)=S$
    \item If there is an isomorphism between $R$ and $S$ we call the rings ``isomorphic" and we denote it by $R\simeq S$.
    \item An isomorphism from $R$ onto itself is also called an ``automorphism".
    \item Note that an automorphism basically permutes the elements of a ring.
    \item However, not every permutation would be an automorphism.
    \item An automorphism is a permutation with structure.
    \end{stepitemize}
\end{frame}

\begin{frame}{Examples}
\begin{stepitemize}
 \item For any ring $R$, the identity map $\phi:R\rightarrow R$ given by $\phi(r)=r$ is a ring isomorphism, or an automorphism.
    \item As we saw before, the identity function is the only automorphism of $\Z$.
    \item Consider $\varphi:\C \rightarrow \C$ by $\varphi(a+bi)=a-bi$,
    \item So, $\varphi$ is the complex conjugation operation.
    \item It is both additive and multiplicative and hence is a ring homomorphism.
    \item Moreover, it is one-to-one and onto since $a-bi=c-di$ implies $a=c$ and $b=d$ and $\varphi(a-bi)=a+bi$.
    \item Thus $\varphi$ is an automorphism of $\C$.
\end{stepitemize}
\end{frame}

\begin{frame}{CRT for Rings}
\begin{stepitemize}
\item Consider $\varphi:\Z_{12}\rightarrow \Z_{3}\times \Z_{4}$, where $\Z_3\times \Z_4$ is the ring with respect to component-wise addition and multiplication modulo $3$ and $4$ respectively.
\item Such rings are called ``product rings".
\item Define $\varphi(m) = (m_3, m_4)$, where $m_3$ and $m_4$ represent the reduction modulo $3$ and modulo $4$, respectively.
\item Since modulo reduction is both multiplicative and additive, this is clearly a ring homomorphism.
\item By CRT, for any $a \in \Z_3$ and $b\in \Z_4$, there is a unique solution to  $x\equiv a \pmod{3}$ and $x\equiv b\pmod{4}$ in $\Z_{12}$.
\item Thus this map is one-to-one and it is also onto.
\item Hence $\varphi$ is an isomorphism.
\end{stepitemize}
\end{frame}

\section{Quotient Rings, Isomorphism Theorems}

\begin{frame}{Quotient Ring}
\begin{stepitemize}
    \item For a ring $R$ and an ideal $I$ of $R$, we consider the set of additive cosets:
$$R/I = \{a+I|a\in R.\}$$
\item Recall that $(R/I,+)$ is an Abelian group, where we consider the addition in cosets as: $(a+I)+(b+I) = (a+b)+I$.
\item To turn $R/I$ into a ring, we need to introduce the multiplication as well.
\item Define $(a+I)(b+I)=ab+I$.
\item This operation would not be well defined if $I$ were a subring but not an ideal.
\item With these, $R/I$ turns into a ring with the additive identity being $0+I$ or $I$ and the multiplicative identity being $1+I$.
\item All other properties of a ring are satisfied as well. This is called a ``quotient ring".
\end{stepitemize}
\end{frame}

\begin{frame}{Example}
\begin{stepitemize}
    \item Let $R=\Z$ and $I=3\Z$.
    \item Then as before, we have
    $$R/I = \{3\Z, 1+3\Z, 2+3\Z\},$$
\item but now this set has the additional operation of multiplication defined on it by
    $$(1+3\Z)(2+3\Z)=2+3\Z, \:\:\:\: (2+3\Z)(2+3\Z) = 4+3\Z=1+3\Z, ...$$
\end{stepitemize}
\end{frame}

\begin{frame}{Isomorphism Theorems}
\begin{stepitemize}
    \item Consider The function $\phi:\Z/3\Z\rightarrow \Z_3$ given by $\phi(a+3\Z)=a$, which is an isomorphism.
    \item Let us look at the operation tables

\begin{columns}
        \begin{column}{0.4\textwidth}
     \begin{table}
\begin{tabular}{ c| c | c | c}
+ & {\color{blue}$0+3\Z$} & {\color{red} $1+3\Z$} & {\color{teal} $2+3\Z$} \\
\hline
{\color{blue} $0+3\Z$} & {\color{blue} $0+3\Z$} & {\color{red} $1+3\Z$} & {\color{teal} $2+3\Z$}  \\
\hline
{\color{red} $1+3\Z$} & {\color{red} $1+3\Z$} & {\color{teal} $2+3\Z$} & {\color{blue} $0+3\Z$}  \\
\hline
{\color{teal} $2+3\Z$} & {\color{teal} $2+3\Z$} & {\color{blue} $0+3\Z$} & {\color{red} $1+3\Z$} \\
\hline
\end{tabular}
\end{table}
    \end{column}

        \begin{column}{0.4\textwidth}
    \begin{table}
\begin{tabular}{ c| c | c | c}
+ & {\color{blue} $0$} & {\color{red} $1$} & {\color{teal} $2$} \\
\hline
{\color{blue} $0$} & {\color{blue} $0$} & {\color{red} $1$} & {\color{teal} $2$}  \\
\hline
{\color{red} $1$} & {\color{red} $1$} & {\color{teal}$2$} & {\color{blue} $0$}  \\
\hline
{\color{teal} $2$} & {\color{teal} $2$} & {\color{blue} $0$} & {\color{red} $1$} \\
\hline
\end{tabular}
\end{table}
    \end{column}
\end{columns}

\item[]
\begin{columns}
        \begin{column}{0.4\textwidth}
     \begin{table}
\begin{tabular}{ c| c | c | c}
$\cdot$  & {\color{blue} $0+3\Z$} & {\color{red} $1+3\Z$} & {\color{teal}$2+3\Z$} \\
\hline
{\color{blue} $0+3\Z$} & {\color{blue} $0+3\Z$} & {\color{blue} $0+3\Z$} & {\color{blue} $0+3\Z$}  \\
\hline
{\color{red} $1+3\Z$} & {\color{blue}$0+3\Z$} & {\color{red} $1+3\Z$} & {\color{teal}$2+3\Z$}  \\
\hline
{\color{teal} $2+3\Z$} & {\color{blue} $0+3\Z$} & {\color{teal}$2+3\Z$} & {\color{red} $1+3\Z$} \\
\hline
\end{tabular}
\end{table}
    \end{column}

        \begin{column}{0.4\textwidth}
    \begin{table}
\begin{tabular}{ c| c | c | c}
$\cdot$ & {\color{blue} $0$}& {\color{red} $1$} & {\color{teal} $2$} \\
\hline
{\color{blue} $0$} & {\color{blue}$0$} & {\color{blue}$0$} & {\color{blue}$0$}  \\
\hline
{\color{red} $1$} & {\color{blue} $0$} & {\color{red} $1$} & {\color{teal} $2$}  \\
\hline
{\color{teal} $2$} & {\color{blue} $0$} & {\color{teal} $2$} & {\color{red} $1$} \\
\hline
\end{tabular}
\end{table}
    \end{column}
\end{columns}

\end{stepitemize}
\end{frame}

\begin{frame}{1st Isomorphism Theorem}
    \begin{stepitemize}
        \item The isomorphism theorems will formalize this phenomenon:
\begin{theorem}$($ 1st Isomorphism Theorem for Rings $)$
Let $\varphi: R\rightarrow S$ be a homomorphism. Then we have
$$R/ker(\varphi) \simeq ran(\varphi).$$
In particular, if $\varphi$ is onto, then
$$R/ker(\varphi) \simeq S. $$
\end{theorem}
    \end{stepitemize}
\end{frame}

\begin{frame}{Examples}
    \begin{stepitemize}
        \item We can formalize the above isomorphism by considering the ring homomorphism $\varphi:\Z\rightarrow \Z_3$ given by $\varphi(m) = (m)_3$, which is onto and has kernel $3\Z$.
        \item Thus by 1st isomorphism theorem we have
    $$\Z/3\Z \simeq \Z_3.$$
    \item Consider $\phi:\Z[x]\rightarrow \Z$ given by $\phi(p(x)) = p(0)$.
    \item Then $\phi$ is a ring homomorphism that is onto and the kernel is given by
    $$ker(\phi) = \{p(x) \in \Z[x]|p(0)=0\} = x\Z[x] = (x).$$
    \item Hence by 1st isomorphism theorem we have
    $$\Z[x]/(x) \simeq \Z.$$
\item This isomorpshism can also be interpreted as follows.
\item Taking $\Z[x]/(x)$ amounts to reducing polynomials modulo $x$, or ``killing" the $x$-terms, as a result of which we are left with constants, or the ring $\Z$ itself.
    \end{stepitemize}
\end{frame}
\begin{frame}{Another Example}
\begin{stepitemize}
    \item Now let us define a similar homomorphism as the previous one with one slight modification:
$$\phi:\Z[x]\rightarrow \Z_2, \:\:\:\: \phi(x) = (\phi(0))_2,$$
\item that is we reduce $p(0)$ modulo $2$.
\item Similar to the previous case, $\phi$ is a homomorphism that is onto.
\item Kernel consists of all polynomials whose constant terms are even, which is an ideal that is generated by $(x,2)$.
\item One way to see this is to reach the image, we need to ``kill" the $x$-term as well as $2$, to get a $0$ or $1$ in the end.
\item Using the first isomorphism theorem now we have the following isomorphism of rings:
    $$\Z[x]/(x,2) \simeq \Z_2.$$
\end{stepitemize}
\end{frame}
\section{Maximal Ideals}
\begin{frame}{Maximal Ideals}
    \begin{stepitemize}
        \item Let $R$ be a commutative ring with identity and let $M$ be a proper ideal (i.e., $M \neq R$).
        \item $M$ is called a ``maximal ideal" if for an ideal $I$, $M\subseteq I$ implies $I=M$ or $I=R$.
\item So there can be no non-trivial ideal between $I$ and $R$.
\item  $M$ is a maximal ideal if and only if $(x,M)=R$ for every $x\not \in M$.
\item Here $(x,M)$ represents the ideal obtained by ``attaching" $x$ to $M$, that is
$$(x,M) = \{rx+m|r\in R, m\in M\}.$$
\item Let $R$ be a commutative ring with identity. An ideal $M$ or $R$ is maximal if and only if $R/M$ is a field.
\item So, in a sense maximal ideals can help us construct fields.
\item We will see this later in the advanced topics where we will observe how this is used in BGV.
\end{stepitemize}
\end{frame}

\begin{frame}{Examples}
\begin{stepitemize}
    \item In $\Z$ the ideals $(p)$, where $p$ is a prime are all maximal. This can also be seen by the isomorphism theorems:
    $$\Z/(p)\simeq \Z_p,$$
    which are all fields when $p$ is prime.
    \item In $\Z$, the ideal $(4)$ is not maximal for example as
    $$(4) \subsetneq (2) \subsetneq \Z.$$
    \item If $F$ is a field, then the only ideals of $F$ are $0$ and $F$ itself.
    \item This implies that the zero ideal $\{0\}$ is a maximal ideal in $F$.
    \item In fact we can say that a ring $F$ is a field if and only if the zero ideal is maximal in $F$.
    \item Thus, in $\Z_p, \Q, \R, \C$, the only maximal ideal is $\{0\}$.
\end{stepitemize}
\end{frame}
\begin{frame}{More Examples}
\begin{stepitemize}
    \item The ideal $(x)$ is not maximal in $\Z[x]$ since (as we saw in the previous section) $\Z[x]/(x)\simeq \Z$, which is not a field.
    \item Another way to see this is to notice that
    $$(x) \subsetneq (x,2) \subsetneq \Z[x].$$
    \item However, $(x)$ is maximal in $\R[x]$ as by the same reason, $\R[x]/(x)\simeq \R$, which is a field.
    \item $(x)$ is maximal in $\Q[x]$ and $\C[x]$ as well.
    \item Since (as we saw before) $\Z[x]/(2,x) \simeq \Z_2$ is a field, we can say that $(2,x)$ is a maximal ideal of $\Z[x]$.
    \item Similarly $(p,x)$ is a maximal ideal of $\Z[x]$ for all primes $p$.
\end{stepitemize}
\end{frame}

\begin{frame}{Even more examples}
\begin{stepitemize}
        \item $2\Z_4$ is a maximal ideal of $\Z_4$.
        \item Generally, maximal ideals of $\Z_n$ are generated by $p$, where $p$ is a prime divisor of $n$.
        \item In general, all ideals of $\Z_n$ are generated by positive divisors of $n$,
        \item while the maximal ideals are generated by the prime divisors.
    \item If $p$ is a prime number, then $\Z_{p^n}$ has only one maximal ideal given by $(p) = p\Z_{p^n}$.
    \item Such rings that have a unique maximal ideal are called ``local rings".
\end{stepitemize}
\end{frame}
\section{Fields:A General Overview}
\begin{frame}{Overview of Fields}
\begin{stepitemize}
    \item a field $F$ has two operations: $+$ and $\cdot$ with the following properties:
\begin{enumerate}
    \item $a+b\in F$ and $a\cdot b \in F$ for all $a,b\in F$
    \item $a+b=b+a, ab=ba$ for all $a,b\in F$
    \item $a+(b+c)=(a+b)+c$ and $a(bc) = (ab)c$ for all $a,b,c \in F$
    \item $a(b+c)=ab+ac$, for all $a,b,c \in F$
    \item $0\in F$ such that $0+a=a$ for all $a\in F$
    \item $1\in F$ such that $1\cdot a=a$ for all $a\in F$
    \item For all $a\in F$, $-a \in F$ such that $a+(-a)=0$
    \item For all $a \in F$, there exists $a^{-1}\in F$ such that $a\cdot a^{-1}=1$.
\end{enumerate}
\end{stepitemize}
\end{frame}

\begin{frame}{Observation and examples}
\begin{stepitemize}
    \item If  $F$ is a field then $F^{\times} = F-\{0\}$ is a group under $\cdot$ operation.
\item As we saw before, $\Z_p:$ $p$ prime, $\Q, \R, \C$ are all fields.
\end{stepitemize}
\end{frame}

\begin{frame}{Another Example}
\begin{stepitemize}
    \item Let $R = \Z_2[\omega]$, where $\omega^2=\omega+1$. \item We first note that in $\Z_2$, $1$ and $-1$ are the same.
    \item Since $\omega^2=\omega+1$, the ring $\Z_2[w]$ consists of four elements:
    $$\Z_2[\omega] = \{0, 1, \omega, 1+\omega\}.$$
    \item We can actually write the addition and multiplication table to get a more clear picture:
    \begin{columns}
        \begin{column}{0.5\textwidth}
     \begin{table}
\begin{tabular}{ c| c | c |c|c}
$+$  & $0$ & $1$ & $\omega$ & $1+\omega$ \\
\hline
$0$ & $0$ & $1$ & $\omega$ & $1+\omega$  \\
\hline
$1$ & $1$ & $0$ & $1+\omega$ & $\omega$  \\
\hline
$\omega$ & $\omega$ & $1+\omega$ & $0$ & $1$ \\
\hline
$1+\omega$ & $1+\omega$& $\omega$ & $1$&$0$\\
\end{tabular}
\end{table}
    \end{column}

        \begin{column}{0.4\textwidth}
    \begin{table}
\begin{tabular}{ c| c | c |c|c}
$\cdot$  & $0$ & $1$ & $\omega$ & $1+\omega$ \\
\hline
$0$ & $0$ & $0$ & $0$ & $0$  \\
\hline
$1$ & $0$ & $1$ & $\omega$ & $1+\omega$  \\
\hline
$\omega$ & $0$ & $\omega$ & $1+\omega$ & $1$ \\
\hline
$1+\omega$ & $0$& $1+\omega$ & $1$&$\omega$\\
\end{tabular}
\end{table}
    \end{column}
\end{columns}
\item As can be seen, every non-zero element has an inverse with respect to multiplication.
\item This means that $\Z_2[w]$ is a field of size $4$
    \end{stepitemize}
    \end{frame}

\begin{frame}{Characteristic of a Field}
    \begin{stepitemize}
        \item In the example $\Z_2[\omega]$, we have $\omega+\omega=0$.
        \item In fact for any element $x \in \Z_2[\omega]$, we have $x+x=0$.
        \item This is related to the concept of the ``characteristic" of a field.
        \item The ``characteristic" of $R$, denoted $char(R)$ is the order of $1$ in the additive group $(R,+)$ if the order is finite.
        \item If not, then characteristic is $0$.
\item In other words, the characteristic is the smallest positive integer $n$ such that $n1 =0$.
\item If such a positive integer does not exist, then we say that the characteristic is $0$.
    \end{stepitemize}
\end{frame}

\begin{frame}{Examples}
\begin{stepitemize}
   \item $char(\Z_n)=n$ for all $n\geq 2$.
    \item $char(\Z), char(\Q), char(\R), char(\C)$ are all $0$.
   \item $char(\Z_2\times \Z_2) = 2$
    \item In general $char(\Z_m\times \Z_n) = LCM(m,n)$.
\item If $F$ is a field, then $char(F)=0$ or is prime.
\end{stepitemize}
\end{frame}

\section{Field Automorphisms}
\begin{frame}{Field Automorphisms}
\begin{stepitemize}
\item A field automorphism is an isomorphism $\phi:F\rightarrow F$.
\item Such an automorphism satisfies the following properties:
\begin{enumerate}
    \item $\phi(0)=0$
    \item $\phi(1)=1$
    \item $\phi(nx)=n\phi(x)$, for all $x\in F$ and positive integers $n$.
    \item $\phi(x^n)=\phi(x)^n$ for all $x\in F$ and integers $n$.
    \item If $F$ is a finite field, then $\phi$ will actually be a permutation.

\end{enumerate}
\item We already saw some examples of automorphisms, such as $\phi:\C\rightarrow \C$ given by $\phi(a+bi)=a-bi$, i.e., the conjugation operation
\item and the trivial automorphism of $\phi(x)=x$ which works for any field.
\end{stepitemize}
\end{frame}

\begin{frame}{A new Example}
\begin{stepitemize}
    \item Let $\phi:\Z_2[\omega]\rightarrow \Z_2[\omega]$ be given by $\phi(x)=x^2$.
    \item We will demonstrate that this is a field automorphism.

\item The first step is to prove that it is a homomorphism. \item The square function is already multiplicative.
\item To prove that it is additive, we just use the fact that the characteristic is $2$, and hence we have
$$\phi(x+y) = (x+y)^2=x^2+2xy+y^2=x^2+y^2 = \phi(x)+\phi(y)$$
\item since $2xy=0$ in $\Z_2[\omega]$.
\item Thus $\phi$ is a homomorphism.
\item To see that it is a bijection, we just need to look at $\phi(\{0,1,\omega, 1+\omega\}) = \{0,1, 1+\omega, \omega\}$.

\end{stepitemize}
\end{frame}

\begin{frame}{Freshman's Dream}
\begin{stepitemize}
    \item We can generalize the previous example by introducing the well-known ``Freshman's Dream" theorem:
\begin{theorem}$($ Freshman's Dream $)$
If $F$ is a field of characteristic $p$, then
$$(a+b)^p = a^p+b^p,  \:\:\:\forall a,b \in F.$$
\end{theorem}
\item Just by applying the theorem several times, and observing that $(a+b)^{p^2} = ((a+b)^p)^p$, we get the following corollary:
\begin{corollary}
If $F$ is a field of characteristic $p$, then
$$(a+b)^{p^k}= a^{p^k}+b^{p^k},  \:\:\:\forall a,b \in F, k\geq 0.$$
\end{corollary}
\end{stepitemize}
\end{frame}

\begin{frame}{Frobenius Automorphism}
    \begin{stepitemize}
    \item The above corollary leads to an important class of automorphisms for fields of characteristic $p$,
    \item i.e., the class of Frobenius automorphisms.
    \item In particular, for such a field, we have
$$\sigma_k:F\rightarrow F, \:\:\:\sigma_k(a)=a^{p^k}$$ is an automorphism.
\item Later on, in Galois Field Theory, we will see that in certain cases, these will be all the automorphisms.
\end{stepitemize}
\end{frame}

\section{Polynomial Rings}
\begin{frame}{Polynomial Rings}
\begin{stepitemize}
\item Polynomials are central in many of the applications of HE.
\item We first give a general definition of a polynomial and we will mostly focus on polynomials over special rings.
\item Let $R$ be a commutative ring with identity. The polynomial ring over $R$, denoted by $R[x]$ is given by the set
$$\{a_0+a_1x+\dots +a_nx^n| \: a_i\in R, n\geq 0.\}$$
\item If $p(x)=a_0+a_1x+\dots +a_nx^n$ and $q(x)=b_0+b_1x+\dots+b_mx^m$ where, without loss of generality, we assume $m\geq n$, then we have
$$p(x)+q(x) = (a_0+b_0)+(a_1+b_1)x+\dots +(a_n+b_n)x^n+b_{n+1}x^{n+1}+\dots +b_mx^m$$
\item and $p(x)q(x) = \sum c_kx^k,$
where $c_k = \sum_{i+j=k}a_ib_j$.
\end{stepitemize}
\end{frame}

\begin{frame}{Division in Polynomials}
\begin{stepitemize}
\item We can define division algorithm for polynomials just as we can for integers.
\item However, there is one important difference when we do this.
\item
If we let $p(x)=x^2+1$ and $q(x)=2x$, then we cannot divide $p(x)$ by $q(x)$ in $\Z[x]$.
\item However we can divide $p(x)$ by $q(x)$ in $\Q[x]$:
$$x^2+1 = (2x)(\frac{x}{2})+1,$$ so that the quotient is $x/2$ and the remainder is $1$.
\item What makes a difference here is that $\Q$ is a field but $\Z$ is not.
\end{stepitemize}
\end{frame}

\begin{frame}{}
\begin{stepitemize}
\item Let $F$ be a field and $a(x), b(x) \in F[x]$ be two polynomials.
\item Then there exists polynomials $q(x)$ and $r(x)$ in $F[x]$ such that
$$a(x) = b(x)q(x)+r(x), \:\:\: r(x)=0, \:\:\textrm{or}\:\: deg(r)<deg(b).$$
\item This makes the polynomials ring over field to be a special ring, which is called ``Euclidean Domain".
\item One of the deep implications is that such a ring is a Principal Ideal Domain(PID), namely every ideal is generated by a single element.
\item But can we not really divide polynomials in $\Z[x]$?
\item Yes we can in certain cases
\item For example, if $b(x)$ is a monic polynomial, that is if the leading coefficient of $b(x)$ is $1$, then we can always divide $a(x)$ by $b(x)$ in $\Z[x]$.
\end{stepitemize}
\end{frame}

\section{Irreducible Polynomials}
\begin{frame}{Irreducible polynomials}
\begin{stepitemize}
\item $\Z$ vs $\Z[x]$
\item ``primes" in $\Z$ vs ? in $\Z[x]$
\item Primes in $\Z$ are actually ``irreducible"
\item i.e. they cannot be written as a product of smaller numbers.
\item We can define the same for polynomials
\item Irreducible in $\Z[x]$ and irreducible in $\Q[x]$ maybe different
\item For example $4x+6$ is not irreducible in $\Z[x]$ but it is in $\Q[x]$.
\item {\bf Gauss' Lemma:} For $p(x)=a_o+a_1x+\dots a_nx^n \in \Z[x]$, if
$GCD(a_0, a_1, \dots, a_n)=1$, then $p(x)$ is irreducible in $\Z[x]$ if and only if it is irreducible in $\Q[x]$.
\end{stepitemize}
\end{frame}

\begin{frame}{Why irreducible polynomials?}
\begin{stepitemize}
    \item Why irreducible polynomials?
    \item {\bf Theorem:} If $F$ is a field then $p(x)\in F[x]$ is irreducible if and only if $(p(x))$ is a maximal ideal. \item So irreducible polynomials lead to maximal ideals
    \item and maximal ideals lead to new fields.
    \item Very useful in constructing finite fields, especially.
    \item The same polynomial can have different irreducibility depending on the coefficients
    \item $x^2+1$ is irreducible in $\Z[x], \Q[x]$ but it is not in $\Z_2[x]$ since then it is just the same as $(x+1)^2$.
\end{stepitemize}
\end{frame}

\begin{frame}{Examples}
\begin{stepitemize}
    \item {\bf Theorem:} Let $p(x)\in F[x]$ be a polynomial of degree $2$ or $3$. Then $p(x)$ is irreducible if and only of $p(x)$ does not have any roots in $F$.
\item does not work for higher degrees
\item For example $(x^2+1)^2$ is reducible, but has no roots in $\Q$.
$p(x)=x^2+x+1$ is irreducible in $\Z_2[x]$ since $p(0)=p(1)=1$, which means $p$ has no roots in $\Z_2$.
    \item Both $p(x)=x^3+x+1$ and $q(x)=x^3+x^2+1$ are irreducible in $\Z_2[x]$ since $p(1)=p(0)=q(0)=q(1)=1$ and so they do not have roots in $\Z_2$.
    \item $p(x)=x^2+x+1$ is reducible in $\Z_3[x]$, since $p(1)=0$. In fact it is not hard to see that in $\Z_3[x]$:
    $$x^2+x+1 = x^2-2x+1 = (x-1)^2.$$
    \item On the other hand, $q(x)=x^2+1$ is irreducible in $\Z_3[x]$ since $q(0)=1, q(1)=q(2)=2$.
\end{stepitemize}
\end{frame}

\section{Finite Fields: Constructions, Examples}
\begin{frame}{Prime Fields}
\begin{stepitemize}
    \item Fields constructed from other fields
    \item there is a concept of ``base fields"
    \item A prime finite field is defined to be $\Z_p = \{0, 1,\dots, p-1\}$ for some prime number $p$.
    \item Let $F$ be a finite field of characteristic $p$
    \item Then $1\in F$, $2=1+1\in F$, $\dots$, $p-1=1+1+\dots +1 \in F$
    \item So $\Z_p\subseteq F$.
    \item Moreover, $F$ is a vector space over $\Z_p$.
    \item If its dimension is $k$ then we have $|F| = p^k$.
    \item So all finite fields have prime power size.
\end{stepitemize}
\end{frame}

\begin{frame}{Constructing Finite Fields}
\begin{stepitemize}
    \item To construct a finite field of size $p^n$ we take an irreducible polynomial of degree $n$ over $\Z_p$
    \item Such an irreducible always exists.
    \item Then $\Z_p[x]/(f(x))$ is a finite field of size $p^n$
    \item What are the elements of $\Z_p[x]/(f(x))$?
    \item If $f(x)= x^n+a_{n-1}x^{n-1}+\dots+a_1x+a_0$, then
    \item $1, x, \dots, x^{n-1} \in \Z_p[x]/(x^n)$
    \item $x^n=-a_{n-1}x^{n-1}-\dots-a_1x-a_0$
    \item So $\{1, x, x^2, \dots, x^{n-1}\}$ forms a basis for $\Z_p[x]/(f(x))$.
    \item Every polynomial in $\Z_p[x]/(f(x))$ is of degree at most $n-1$.
\end{stepitemize}
\end{frame}

\begin{frame}{Examples}
\begin{stepitemize}
\item The example we saw above, namely $\{0,1,\omega, 1+\omega\}$ is just $\Z_2[x]/(x^2+x+1)$.
    \item Now let us construct a finite field of size $8$.
    \item $x^3+x+1$ is irreducible over $\Z_2$.
    \item So we can construct a finite field of size $8$ by letting $F= \Z_2[x]/(x^3+x+1)$.

    \item So, letting $\omega = x+(x^3+x+1)$, we see that $\omega^3=\omega+1$ and thus we have
    $$F = \{0,1, \omega, \omega ^2, 1+\omega, 1+\omega^2, \omega+\omega^2, 1+\omega+\omega^2\}.$$
    \item How do we add and multiply?
    \item Addition is done just like in a vector space. \item However, when multiplying, we need to remember that in this field $\omega^3=\omega+1$, $\omega^4=\omega^2+\omega$, $\omega^5 = \omega^3+\omega^2= \omega^2+\omega+1$, etc.
\end{stepitemize}
\end{frame}

\begin{frame}{Addition Table of $\Z_2[x]/(x^2+x+1)$}

\begin{table}[]
\tiny{
\begin{tabular}{ p{8mm}|p{8mm}|p{8mm} |p{8mm}|p{8mm}|p{8mm}|p{8mm}|p{8mm}|p{8mm}}
$+$  & $0$ & $1$ & $\omega$ & $\omega^2$ & $1+\omega$ & $1+\omega^2$ & $\omega+\omega^2$ & $1+\omega+\omega^2$ \\
\hline
$0$ &$0$ & $1$ & $\omega$ & $\omega^2$ & $1+\omega$ & $1+\omega^2$ & $\omega+\omega^2$ & $1+\omega+\omega^2$   \\
\hline
$1$ & $1$ & $0$ & $1+\omega$ & $1+\omega^2$ & $\omega$ & $\omega^2$ & $1+\omega+\omega^2$ & $\omega+\omega^2$  \\
\hline
$\omega$ & $\omega$ & $1+\omega$ & $0$ & $\omega+\omega^2$  & $1$ & $1+\omega+\omega^2$ & $\omega^2$ & $1+\omega^2$\\
\hline
$\omega^2$ & $\omega^2$& $1+\omega^2$ & $\omega+\omega^2$&$0$ & $1+\omega+\omega^2$ &$1$& $\omega$ & $ 1+\omega$\\
\hline
$1+\omega$ & $1+\omega$& $\omega$ & $1$&$1+\omega+\omega^2$ & $0$ &$\omega+\omega^2$& $1+\omega^2$ & $\omega^2$\\
\hline
$1+\omega^2$ & $1+\omega^2$& $\omega^2$&$1+\omega+\omega^2$ & $1$&$\omega+\omega^2$ &$0$& $1+\omega$ & $\omega$\\
\hline
$\omega+\omega^2$ & $\omega+\omega^2$&$1+\omega+\omega^2$& $\omega^2$&$\omega$ & $1+\omega^2$&$1+\omega$ &$0$& $1$\\
\hline
$1+\omega+\omega^2$ & $1+\omega+\omega^2$&$\omega+\omega^2$& $1+\omega^2$&$1+\omega$ & $\omega^2$&$\omega$ &$1$& $0$\\

\end{tabular}
}
\end{table}
\end{frame}

\begin{frame}{Multiplication Table of $\Z_2[x]/(x^2+x+1)$}
\begin{table}[H]
\tiny{
\begin{tabular}{ p{8mm}|p{8mm}|p{8mm} |p{8mm}|p{8mm}|p{8mm}|p{8mm}|p{8mm}|p{8mm}}
$\cdot$  & $0$ & $1$ & $\omega$ & $\omega^2$ & $1+\omega$ & $1+\omega^2$ & $\omega+\omega^2$ & $1+\omega+\omega^2$ \\
\hline
$0$ &$0$ & $0$ & $0$ & $0$ & $0$ & $0$ & $0$ & $0$   \\
\hline
$1$ & $0$ & $1$ & $\omega$ & $\omega^2$ & $1+\omega$ & $1+\omega^2$ & $\omega+\omega^2$ & $1+\omega+\omega^2$ \\
\hline
$\omega$ & $0$ & $\omega$ & $\omega^2$ & $1+\omega$  & $\omega+\omega^2$ & $1$ & $1+\omega+\omega^2$ & $1+\omega^2$\\ \hline
$\omega^2$ & $0$ & $\omega^2$ & $1+\omega$&$\omega+\omega^2$ & $1+\omega+\omega^2$ &$\omega$& $1+\omega^2$ & $1$\\
\hline
$1+\omega$ & $0$& $1+\omega$ & $\omega+\omega^2$&$1+\omega+\omega^2$ & $1+\omega^2$ &$\omega^2$& $1$ & $\omega$\\
\hline
$1+\omega^2$ & $0$& $1+\omega^2$&$1$ & $\omega$&$\omega^2$ &$1+\omega+\omega^2$& $1+\omega$ & $\omega+\omega^2$\\
\hline
$\omega+\omega^2$ & $0$&$\omega+\omega^2$& $1+\omega+\omega^2$&$1+\omega^2$ & $1$&$1+\omega$ &$\omega$& $\omega^2$\\
\hline
$1+\omega+\omega^2$ & $0$&$1+\omega+\omega^2$& $1+\omega^2$&$1$ & $\omega$&$\omega+\omega^2$ &$\omega^2$& $1+\omega$\\
\end{tabular}
}
\end{table}

\end{frame}

\begin{frame}{Another Example}
    \begin{stepitemize}
    \item We can construct a field $F$ of size $9$ by letting
$$F = \Z_3[x]/(x^2+1) = \{a+b\omega|a, b\in \Z_3, \:\omega^2=-1\}.$$

\item The multiplication and addition will work very much like they would in the case of complex numbers,
\item the only difference is that the operations between integers is done modulo $3$:
\item
$(a+b\omega)+(c+d\omega) = (a+c)+(b+d)\omega,$
\item $(a+b\omega)\cdot(c+d\omega)=(ac-bd)+(ad+bc)\omega.$

    \end{stepitemize}
\end{frame}
\begin{frame}
\centerline{ \color{blue} \bf{\large THANK YOU!!!}}

\end{frame}

\end{document}
