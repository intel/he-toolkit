
\documentclass[12pt]{article}
% Standard ams packages
\usepackage{amsmath, amssymb, amsthm, graphicx}
\usepackage{float}
% Edit margins
\usepackage[letterpaper, margin=1in, left=0.8in]{geometry}
\usepackage{verbatim}
% Resume enumeration after a break
\usepackage{enumitem}
\makeatletter
\def\verbatim@nolig@list{\do\`\do\<\do\>\do\'\do\-}% no comma
\makeatother
%\pagenumbering{gobble}
\usepackage{tikz}
% Define macros
\global\long\def\dom{\mathop\mathrm{dom}\nolimits}   % domain
\global\long\def\Ker{\mathop\mathrm{Ker}\nolimits} % kernel
\global\long\def\Im{\mathop\mathrm{Im}\nolimits} % image
\global\long\def\C{\mathbb{C}}                       % complex
\global\long\def\R{\mathbb{R}}                       % reals
\global\long\def\Q{\mathbb{Q}}                       % rationals
\global\long\def\Z{\mathbb{Z}}                       % integers
\global\long\def\N{\mathbb{N}}                      % naturals
\global\long\def\A{\mathcal{A}}
\global\long\def\F{\mathbb{F}}
\def\div{\, \big| \,} % divides
\def\inv{^{-1}} % inverse
\def\tr{\text{Trace}} % trace
\def\GL{\text{GL}} % general linear
\def\SL{\text{SL}} % special linear
\def\char{\text{char}} % characteristic

% Generator of a group
\newcommand{\gen}[1]{\langle #1 \rangle}
\renewcommand{\qedsymbol}{\(\blacksquare\)}

\theoremstyle{plain}
\newtheorem{corollary}{Corollary}
\newtheorem{lemma}{Lemma}
\newtheorem{example}{Example}
\newtheorem{observation}{Observation}
\newtheorem{proposition}{Proposition}
\newtheorem{theorem}{Theorem}
\newtheorem{axiom}{Axiom}
\newtheorem{question}{Question}

\theoremstyle{definition}
\newtheorem{definition}{Definition}

\theoremstyle{remark}
\newtheorem{remark}{Remark}

% Quick permutation group notation (3 elements)
\newenvironment{permutation3}
{
\left(\begin{tabular}{ccc}
}
{
\end{tabular}\right)
}

% Quick permutation group notation (4 elements)
\newenvironment{permutation4}
{
\left(\begin{tabular}{cccc}
}
{
\end{tabular}\right)
}

\newcommand{\ext}[1]{% a one shot command for this display
  \hphantom{\scriptstyle#1}\bigg|{\scriptstyle#1}%
  }
% Quick permutation group notation (5 elements)
\newenvironment{permutation5}
{
\left(\begin{tabular}{ccccc}
}
{
\end{tabular}\right)
}

% Quick permutation group notation (6 elements)
\newenvironment{permutation6}
{
\left(\begin{tabular}{cccccc}
}
{
\end{tabular}\right)
}

% Quick permutation group notation (7 elements)
\newenvironment{permutation7}
{
\left(\begin{tabular}{ccccccc}
}
{
\end{tabular}\right)
}

\title{Galois Theory and Cyclotomics}
\author{Bahattin Yildiz }
\begin{document}
\date{}

\maketitle
In this chapter, we will be talking about some of topics that appear heavily in implementations of Homomorphic Encryption. More precisely we will be talking abaout Galois Theory and Cyclcotimcs, both of which play crucial roles in ``slot structure" and ``slot algebras". In Galois Theory we will be talking about the automorphisms of the fields, in particular of the finite fields, which are generated by the Frobenius automorphism. We will in particular see how Frobenius automorphisms allow permuting the slots. In the second part we will talk about cyclotomics and see how they lead to the formation of equi-sized slots. We assume that the reader has already familiarized themselves with the previous material on Number Theory, Groups and Rings/Fields as we will be building up on the material mentioned here. 


 \section{Galois Theory of Fields: Extensions, Field automorphisms}
We start with the concept of a subfield and extensions:
\subsection{Subfields and Extensions}
\begin{definition}
Let $K$ and $F$ be two fields with $F\subseteq K$. We then say ``$K$ is an extension of $F$" or ``$F$ is a subfield of $K$". We denote this relation by $K:F$. 
Diagramatically, this will be shown by:
\[
\begin{array}{@{}c@{}}
K\\
\ext{} \\
F \\
\end{array}
\]

\end{definition}
\begin{example}
\begin{enumerate}
    \item $\R$ is a extension of $\Q$ and $\C$ is an extension of $\R$. 
    The diagram shows this relation:
\[
\begin{array}{@{}c@{}}
\mathbb{C}\\
\ext{} \\
\mathbb{R} \\
\ext{} \\
\mathbb{Q}
\end{array}
\]

\item $\Q(\sqrt{2})$ is defined as the subring of $\R$ that consists of ``adjoining" $\sqrt{2}$ to $\Q$. Namely, 
$$\Q(\sqrt{2})=\{a+b\sqrt{2}|a,b\in \Q\}.$$
It is a field since if $a+b\sqrt{2}\neq 0$, then  
$$\frac{1}{a+b\sqrt{2}} = \frac{a-b\sqrt{2}}{a^2-2b^2} = \frac{a}{a^2-2b^2}+\frac{-b}{a^2-2b^2}\sqrt{2} \in \Q(\sqrt{2}).$$
$\Q(\sqrt{2})$ contains $\Q$ as a subfield and is itself a subfield of $\R$. Thus we have the following diagram:

\[
\begin{array}{@{}c@{}}
\mathbb{R}\\
\ext{} \\
\mathbb{Q}(\sqrt{2}) \\
\ext{} \\
\mathbb{Q}
\end{array}
\]
\item As we saw before in finite fields, $\F_{p^k}$ contains $\F_p=\Z_p$ as a subfield, and hence is a field extension of the prime field $\F_p$:
\[
\begin{array}{@{}c@{}}
\mathbb{F}_{p^k}\\
\ext{} \\
\mathbb{F}_p \\
\end{array}
\]
\end{enumerate}
\end{example}

The following theorem brings us back to the prime subfields:
\begin{theorem}
Any field of characteristic $0$ is an extension field of $\Q$, while any field of characteristic $p$ is an extension field of $\Z_p$. 
\end{theorem}
One way to see this is that a characteristic $0$ field contains $1$ and hence it must contain $1+1=2, 1+1+1=3, -1, -1-1=-2,$ etc. and since it is a field, it must contain $1/2$, $-1/3$, etc. Hence any characteristic $0$ field must contain $\Q$ as a subfield. (A more accurate statement would be to say that ``every characteristic $0$ field contains an isomorphic copy of $\Q$ as a subfield", but we do not really distinguish between isomorphic fields. )

A related important theorem, that forms the basis of Galois Theory is the following:
\begin{theorem}
Let $\sigma$ be any automorphism of a field $F$. If $char(F)=p$, then $\sigma$ is the identity on $\Z_p$ and if $char(F)=0$, then $\sigma$ is the identity on $\Q$.
\end{theorem}
\begin{proof}
We just sketch the idea of the proof. Assuming $char(F)=0$ and $\sigma:F\rightarrow F$ is an automorphism, then we have 
$\sigma(0)=0$ and $\sigma(1)=1$. But then $\sigma(2)=\sigma(1+1)=\sigma(1)+\sigma(1)=2.$ This way we can easily see that $\sigma(n)=n$ for all $n\in \Z$. Now for the rationals, we can do the following:
$$1=\sigma(1) = \sigma\left (\underbrace{\frac{1}{n} +\frac{1}{n} +\cdots + \frac{1}{n}}_{n\rm\ times}\right ) = \sigma\left (\frac{1}{n}\right )+\cdots+\sigma \left (\frac{1}{n}\right ) = n\cdot\sigma\left (\frac{1}{n}\right ), $$
from which we get $\sigma\left (\frac{1}{n}\right ) = \frac{1}{n}$. Finally, because $\sigma$ is a ring homomrphism, we can write 
$$\sigma\left (\frac{m}{n}\right ) = \sigma(m)\sigma\left (\frac{1}{n}\right ) = m\cdot \frac{1}{n}=\frac{m}{n}. $$
\end{proof}

The theorem gives us a powerful tool to find all automorphisms of some fields:
\begin{example}
Let us find all automorphisms of $\Q(\sqrt{2})$. Assuming $\sigma$ is one such automorphism, we know, from the previous theorem, that $\sigma(q)=q$ for all $q\in \Q$. Using this, let us focus on $\sigma(\sqrt{2})$. 
Since,
$$2=\sigma(2) = \sigma (\sqrt{2}\cdot \sqrt{2}) = \sigma(\sqrt{2})\cdot \sigma(\sqrt{2}) = \sigma(\sqrt{2})^2,$$
we see that $\sigma(\sqrt{2})=\sqrt{2}$ or $-\sqrt{2}$. 
Thus we have two cases:
\begin{enumerate}
    \item If $\sigma(\sqrt{2})=\sqrt{2}$, then because it is a ring homomorphism, we get 
    $$\sigma(a+b\sqrt{2})=\sigma(a)+\sigma(b)\sigma(\sqrt{2})=a+b\sqrt{2},$$ namely it is the identity automorphism. 
    \item If $\sigma(\sqrt{2})=-\sqrt{2}$, then we get 
    $$\sigma(a+b\sqrt{2}) = a-b\sqrt{2}.$$
\end{enumerate}
Since automorphisms are functions from $F$ to $F$, we can actually compose them. Note that for the automorphism $\sigma(a+b\sqrt{2}) = a-b\sqrt{2}$, we have 
$$\sigma(\sigma(a+b\sqrt{2})) = \sigma(a-b\sqrt{2}) = a-(-b)\sqrt{2}=a+b\sqrt{2},$$
meaning that $\sigma^2$ actually gives us the identity map. 
This kind of idea will be present in most of what we do in Galois theory.
\end{example}

\begin{example}
We can sometimes adjoin more than one element to the same field. For example $\Q(\sqrt{2}, \sqrt{3})$ is a field where each element consists of elements of the form 
$$a+b\sqrt{2}+c\sqrt{3}+d\sqrt{6}, \:\:\:a,b,c,d\in \Q.$$
We can see some intermediary fields and this time the diagram looks like the following:

    
    \begin{tikzpicture}
    \node (Q1) at (0,0) {$\mathbb{Q}$};
    \node (Q2) at (2,2) {$\mathbb{Q}\Big(\sqrt{3}\Big)$};
    \node (Q3) at (0,4) {$\mathbb{Q}\Big(\sqrt{2} \, , \sqrt{3}\Big)$};
    \node (Q4) at (-2,2) {$\mathbb{Q}\Big(\sqrt{2}\Big)$};

    \draw (Q1)--(Q2);
    \draw (Q1)--(Q4);
    \draw (Q3)--(Q4);
    \draw (Q2)--(Q3);
        \end{tikzpicture}
\end{example}

Let us finish this section by giving out some exercises:
\begin{enumerate}
    \item Find two examples of fields $K$ such that $\Q \subsetneq K \subsetneq \Q(\sqrt{2},i)$. Note that these are subfields of $\C$.
    \item Find the subfields of $\C$ generated by $7/8$, $2+3i$, by $\R \cup \{i\}$.
\end{enumerate}

\subsection{Simple Extensions}
The main idea here is the following: Given a field $F$ and a polynomial $f(x) \in F[x]$, can we find an extension $K$ of $F$ such that $f$ has a root in $K$? When $F=\Q$, we can simply choose $K=\C$ and it will be done. However, in general it is not going to be this simple. Moreover, even for $\Q$, we might have a ``smaller" field that does the job, maybe we do not need to go as high as $\C$. 

The main idea is to ``adjoin" a root to $F$ as we did before with $\Q$ and $\sqrt{2}$, $\sqrt{3}$, etc. How do we that? Well, we make use of some theory from rings and fields. Let $F$ be a field and $f(x)\in F[x]$ be a monic, irreducible polynomial (we can assume this without loss of generality, since by unique factorization every polynomial can be factorized into irreducibles). Since $f(x)$ is irreducible, the ideal $(f(x))$ is a maximal ideal in $F[x]$. But then $F[x]/(f(x))$ is a field. We can view $F$ as a subfield of $F[x]/(f(x))$ since the $F$ is the field of constants. So let $K=F[x]/(f(x))$. Now let $\alpha = \overline{x}=x+(f(x))\in K$. Then $f(\alpha)=0$ in $K$ because of the quotient ring structure.
\begin{example}
Here are some examples of this idea:
\begin{enumerate}
    \item The field of complex numbers, $\C$, can actually be constructed from $\R$ using this idea as 
    $$\C\simeq \R[x]/(x^2+1).$$
    \item $\Q(\sqrt{2})\simeq \Q[x]/(x^2-2)$
    \item $\F_8\simeq \Z_2[x]/(x^3+x+1)$
    \item The field $\Q[x]/(x^2+1)$ is an interesting field as it is the field of Gaussians. 
\end{enumerate}
\end{example}

\begin{definition}
A field of the form $F(\alpha)$ is called a ``simple" extension of $F$. 
\end{definition}
Many field extensions can be formed in this way. Sometimes an extension that does not look like a simple extension may turn out to be so. For example it is not hard to see that 
$$\Q(\sqrt{2}, \sqrt{3}) =\Q(\sqrt{2}+\sqrt{3}).$$
Indeed, the inclusion $\Q(\sqrt{2}+\sqrt{3}) \subseteq \Q(\sqrt{2}, \sqrt{3})$ is clear. For the reverse inclusion, 
we take 
$$(\sqrt{2}+\sqrt{3})^3 = 2\sqrt{2}+6\sqrt{3}+9\sqrt{2}+3\sqrt{3} = 11\sqrt{2}+9\sqrt{3} \in \Q(\sqrt{2}+\sqrt{3}).$$
Hence 
$$11\sqrt{2}+9\sqrt{3}-9(\sqrt{2}+\sqrt{3}) = 2\sqrt{2}\in \Q(\sqrt{2}+\sqrt{3}),$$
which implies $\sqrt{2} \in \Q(\sqrt{2}+\sqrt{3})$. 
Finally, 
$$\sqrt{3}=(\sqrt{2}+\sqrt{3})-\sqrt{2} = \sqrt{3}\in \Q(\sqrt{2}+\sqrt{3}).$$

So, what exactly does $F[\alpha]$ look like?
Assuming that $f(x)$ is an irreducible polynomial of degree $d$, we have $1, \alpha, \alpha^2, \dots, \alpha^{d-1}$ being independent but what about $\alpha^d$? Assuming that $f(x)=x^d+a_{d-1}x^{d-1}+\dots +a_1x+a_0$, we see that $f(\alpha)=0$ implies 
$$\alpha^d=-a_{d-1}\alpha^{d-1}-\dots -a_1d-a_0.$$
This shows that the set $\{1, \alpha, \alpha^2, \dots, \alpha^{d-1}\}$ forms a basis for $F(\alpha)$ over $F$
as a vector space. This $d$ then can also be labeled as the ``degree" of the extension and is denoted by $[F(\alpha):F]$. 

The concept of degree is valid for all extensions. That is because any field extension can be viewed as a vector space over the base field. When the dimension is $\infty$, we cannot of course talk about a finite degree. 

What we have shown is that simple extensions all have finite degrees. 

\begin{example}
\begin{enumerate}
 \item What about the degree of $\Q(\sqrt{2}, \sqrt{3})$ over $\Q$ for example? In this case we can see that $\{1, \sqrt{2}, \sqrt{3}, \sqrt{6}\}$ forms a basis so we can say that 
 $$[\Q(\sqrt{2}, \sqrt{3}):\Q]=4.$$
 
 Another way of seeing this is to recall that $\Q(\sqrt{2}, \sqrt{3}) \simeq \Q(\sqrt{2}+\sqrt{3})$. Now letting $\alpha=\sqrt{2}+\sqrt{3}$, we can go ahead and try to find the irreducible polynomial of minimal degree that has $\alpha$ as a root. To do this, we let 
 $\alpha=\sqrt{2}+\sqrt{3}$, which after squaring, gives us 
 $\alpha^2=5+2\sqrt{6}$. Hence $2\sqrt{6}=\alpha^2-5$, which after another squaring, turns into 
 $$\alpha^4-10\alpha^2+1=0.$$
 Thus $f(x)=x^4-10x^2+1$ is a polynomial that has $\sqrt{2}+\sqrt{3}$ as a root. It is not hard to show that $f(x)$ is irreducible over $\Q$. This shows that 
 $$\Q(\sqrt{2}+\sqrt{3}) \simeq \Q[x]/(x^4-10x^2+1)$$ and hence
 $[\Q(\sqrt{2}+\sqrt{3}):\Q]=4$. 
\item $[\C:\R]=2$
\item $[\R:\Q]=\infty$ as $\sqrt{2}, \sqrt{3}, \sqrt{5}, \dots, \sqrt{p}, \dots$ are all independent over $\Q$.  
\item $[\F_8:\F_2]=3$.
\item $[\Q(\sqrt{2}, \sqrt{3}):\Q(\sqrt{2})]=2$ since $x^2-3$ is an irreducible polynomial in $\Q(\sqrt{2})[x]$. 
\end{enumerate}
 \end{example}
 
 \subsection{Automorphism Groups}
 First, let us recall that if $\sigma:F\rightarrow F$ is an automorphism of fields, then 
 \begin{itemize}
     \item $\sigma|_{\Q}$ is the identity if $char(F)=0$
     \item $\sigma|_{\Z_p}$ is the identity if $char(F)=p$. 
 \end{itemize}
We now generalize this concept to field extensions. 
\begin{definition}
Let $F$ and $K$ be fields so that $F\subseteq K$. The set of auotomorphisms of $K$ that ``fix" $F$ is denoted by $Aut(K/F)$. 
\end{definition}
So, we can write
$$Aut(K/F) = \{\sigma|\sigma :K\rightarrow K, \:\:\sigma(a)=a, \:\:\forall a\in F\}.$$
But this ``set" is actually more than just a set.

Let us dig deeper into the structure of $Aut(K/F)$. If $\sigma, \tau \in Aut(K/F)$, then $\sigma\circ\tau = \sigma\tau$ is also an automorphism of $K$ that leaves $F$ fixed. Moreover, the identity auotmorphism $\mathfrak{1}_K$ is also in $Aut(K/F)$. Finally, if $\sigma$ is in $Aut(K/F)$ then the inverse map $\sigma^{-1}$ is also in $Aut(K/F$).Thus we have seen that 
\begin{theorem}
If $K/F$ is an extension field of $F$, then $Aut(K/F)$ is a group under the composition of functions. 
\end{theorem}
\begin{example}
Let us consider some examples of such automorphism groups:
\begin{enumerate}
\item $Aut(\C/\R) = \{\mathfrak{1}, \sigma\}$, where $\sigma(z) = \bar{z}$, is the complex conjugation. Note that $\sigma^2=\mathfrak{1}$, so $Aut(\C/\R)$ is a cyclic group of order $2$.
\item As we saw in the previous section, in a similar way, we have $Aut(\Q(\sqrt{2})/\Q)=\{\mathfrak{1}, \sigma\}$, where 
$\sigma(a+b\sqrt{2})=a-b\sqrt{2}$. As we saw in that example, $\sigma^2=\mathfrak{1}$ and so so $Aut(\C/\R)$ is also a cyclic group of order $2$.
\item Let us consider $Aut(\Q(\sqrt{2}, \sqrt{3})/\Q)$. 
As in the previous example, since any automorphism $\sigma \in Aut(\Q(\sqrt{2}, \sqrt{3})/\Q)$ leaves $\Q$ fixed, we have 
$$\sigma(\sqrt{3})^2=\sigma(\sqrt{3}^2)=\sigma(3)=3$$ and 
$$\sigma(\sqrt{2})^2=\sigma(\sqrt{2}^2)=\sigma(2)=2.$$
From here, we get $\sigma(\sqrt{2})=\pm\sqrt{2}$ and $\sigma(\sqrt{3}) = \pm\sqrt{3}$. Depending on the choice, we actually get $4$ different automorphisms:
$$\mathfrak{1}:\begin{array}{c}
      \sqrt{2} \mapsto \sqrt{2}\\
      \sqrt{3} \mapsto \sqrt{3}
\end{array}, \:\: \mathfrak{\sigma}:\begin{array}{c}
      \sqrt{2} \mapsto -\sqrt{2}\\
      \sqrt{3} \mapsto \sqrt{3}
\end{array}, \:\: \mathfrak{\tau}:\begin{array}{c}
      \sqrt{2} \mapsto \sqrt{2}\\
      \sqrt{3} \mapsto -\sqrt{3}
\end{array}, \:\: \mathfrak{\rho}:\begin{array}{c}
      \sqrt{2} \mapsto -\sqrt{2}\\
      \sqrt{3} \mapsto -\sqrt{3}
\end{array}$$

So $Aut(\Q(\sqrt{2}, \sqrt{3})/\Q) = \{\mathfrak{1}, \sigma, \tau, \rho\}$. However, we can say something more. We see that $$\sigma^2=\tau^2=\rho^2=\mathfrak{1}$$
and it is not hard to see that $\rho = \sigma\tau = \tau\sigma$. So, in this case we see that $Aut(\Q(\sqrt{2},\sqrt{3})/\Q)$ is an Abelian group that is generated by two elements or that 
$$Aut(\Q(\sqrt{2}, \sqrt{3})/\Q) = \{\mathfrak{1}, \sigma, \tau, \sigma\tau\}.$$ This is the smallest non-cyclic group and is also called the ``Klein-$4$ group". 

\item $Aut(\F_{p^k}/\F_p)$ is the cyclic group generated by $\sigma$, where $\sigma $ is the Frobenius automorphism, that is $\sigma(a)=a^p$. So 
$$Aut(\F_{p^k}/\F_p) = \{\mathfrak{1}, \sigma, \sigma^2, \dots, \sigma^{k-1}\}.$$
\end{enumerate}
\end{example}
\subsection{Fixed fields} So, in the previous subsection we saw how we can look at automorphisms of a field extension that leaves a subfield fixed. In this subsection, we will go in the opposite direction. That is, given a field together with an automorphism group, we will consider all the subfields that are fixed by all the auotmorphisms in a certain subgroup:
\begin{definition}
 Assume that $K/F$ is an extension field and let $G=Aut(K/F)$ be the automorphism group. For a subgroup $H$ or $G$, we let the ``fixed field" of $H$ be defined as
 $$Fix(H) = \{\alpha\in K|\sigma(\alpha)=\alpha, \:\:\:\forall \sigma \in H\}.$$
 The best way to see this is through some examples.
 \begin{example}
 \begin{enumerate}
     \item If $G = Aut(K/F)$, then 
     $$Fix(G) = F, \:\:\:\: Fix(\{\mathfrak{1}\})=K.$$
     \item If $G = Aut(\Q(\sqrt{2},\sqrt{3})/\Q)$ and $H_1 = \{\mathfrak{1}, \sigma\}$, $H_2 = \{\mathfrak{1}, \tau\}$ and $H_3=\{\mathfrak{1}, \sigma \tau\}$, then the fixed fields are given as follows:
     $$Fix(H_1) = \Q(\sqrt{3}), \:\:\: Fix(H_2) = \Q(\sqrt{2}), \:\:\: Fix(H_3)=\Q(\sqrt{6}).$$
    This strong connection between the subgroups of $Aut$ and the intermediary extension fields can be seen in the following picture: 
    
     \begin{tikzpicture}
     
    \node (Q1) at (0,0) {$\mathbb{Q}$};
    \node (Q2) at (2,2) {$\mathbb{Q}(\sqrt{6})$};
    \node (Q3) at (0,4) {$\mathbb{Q}(\sqrt{2},\sqrt{3}) $};
    \node (Q4) at (-2,2) {$\mathbb{Q}(\sqrt{2})$};
    \node (Q5) at (0,2) {$\mathbb{Q}(\sqrt{3})$}; 

    \draw (Q1)--(Q2);
    \draw (Q1)--(Q4);
    \draw (Q3)--(Q4);
    \draw (Q5)--(Q3);
    \draw (Q2)--(Q3);
    \draw (Q1)--(Q5);

\node (R1) at (8,0) {$G$};
\node (R2) at (6,2) {$H_3$};
\node (R3) at (8,2) {$H_1$};
\node (R4) at (10,2) {$H_2$};
\node (R5) at (8,4) {$\{\mathfrak{1}\}$};

    \draw (R1)--(R2);
    \draw (R1)--(R4);
    \draw (R1)--(R3);
    \draw (R5)--(R3);
    \draw (R2)--(R5);
    \draw (R4)--(R5);


\draw[dotted] (Q1)--(R1);
\draw[dotted] (Q3)--(R5);
\draw[dotted] (Q2)--(R2);
    \end{tikzpicture}

Note that the diagram of subgroups is sort of ``inverted", but when you look at both diagrams, we see that each field on the left is the fixed field of the corresponding subgroup on the right. 
\end{enumerate}
\end{example}
\end{definition}

\subsection{Galois Theory, Galois extensions}
What we saw just above is what Galois Theory is mainly about. Roughly, Galois theory connects the subgroups of an automorphism group with the intermediary field extensions. 
In what follows, we will formalize this and give a few examples. Note that we will just do a superficial dive into this topic as it is a very deep theory with many applications and we do not need an in depth treatment for our purposes. 

Before moving on, we should make a notation change, namely from here on we will call $Aut(K/F)$ the Galois group of $K$ over $F$ and we will denote it by $Gal(K/F)$.

The following is the fundamental theorem of Galois:
\begin{theorem}
Let $K$ be a ``nice" extension of $F$. (Nice has a mathematical definition that requires more background. It is also called ``a Galois Extension". ) Let
$$\mathcal{E} = \{\textrm{intermediary fields of} \:\: K/F\}$$
and
$$\mathcal{H} = \{\textrm{subgroups of} \:\: Gal(K/F)\}.$$
Then there is a one-to-one correspondence between elements of $\mathcal{E}$ and $\mathcal{H}$. More precisely, we have the following:
\begin{enumerate}
    \item $|Gal(K/E)| = [K:E]$ for all $E\in \mathcal{E}$.
    \item $[K:Fix(H)]=|H|$ for all subgroups $H\in \mathcal{H}$.  
\end{enumerate}
\end{theorem}

We can see this on the following picture:

 \begin{tikzpicture}
     
    \node (Q1) at (0,0) {$F$};
    \node (Q2) at (0,2) {$E_1$};
    \node (Q3) at (0,4) {$E_2$};
    \node (Q4) at (0,6) {$K$};
 
     \draw (Q1)--(Q2);
    \draw (Q2)--(Q3);
    \draw (Q3)--(Q4);


    \node (R1) at (6,0) {$Gal(K,F)$};
    \node (R2) at (6,2) {$Gal(K,E_1)$};
    \node (R3) at (6,4) {$Gal(K,E_2)$};
    \node (R4) at (6,6) {$\mathfrak{1}$};
 
     \draw (R1)--(R2);
    \draw (R2)--(R3);
    \draw (R3)--(R4);

\draw[dotted] (Q1)--(R1);
\draw[dotted] (Q2)--(R2);
\draw[dotted] (Q3)--(R3);
\draw[dotted] (Q4)--(R4);
    \end{tikzpicture}
    
\medskip
\noindent
So, this one-to-one correspondence between the subfields and subgroups is an order-reversing correspondence. 

\begin{example}
\begin{enumerate}
    \item The picture we had in Example 7 is precisely of the same type that we have described for Galois extensions.
\item It is known that $\F_{p^s}$ is a subfield of $\F_{p^t}$ if and only if $s|t$. Recall also that $Gal(\F_{p^s})$ is the cyclic group of order $s$. Let us look at the cases of $\F_{p^8}$ and $\F_{p^6}$. 

\medskip 
\noindent {\bf $\F_{p^8}$:} The intermediary fields in this case are $\F_{p^4}$ and $\F_{p^2}$. The Galois group is $\langle\sigma\rangle$, where $\sigma$ is the Frobenius automorphism that takes $x$ to $x^p$. The order of $\sigma$ is 8. So its intermediary subgroups are $\langle \sigma^2\rangle$ which has order $4$ and $\langle \sigma^4\rangle$, which has order $2$:

\begin{tikzpicture}
     
    \node (Q1) at (0,0) {$\F_p$};
    \node (Q2) at (0,2) {$\F_{p^2}$};
    \node (Q3) at (0,4) {$\F_{p^4}$};
    \node (Q4) at (0,6) {$\F_{p^8}$};
 
     \draw (Q1)--(Q2);
    \draw (Q2)--(Q3);
    \draw (Q3)--(Q4);


    \node (R1) at (6,0) {$\langle\sigma\rangle $};
    \node (R2) at (6,2) {$\langle \sigma^2\rangle$};
    \node (R3) at (6,4) {$\langle \sigma^4\rangle$};
    \node (R4) at (6,6) {$\mathfrak{1}$};
 
     \draw (R1)--(R2);
    \draw (R2)--(R3);
    \draw (R3)--(R4);

\draw[dotted] (Q1)--(R1);
\draw[dotted] (Q2)--(R2);
\draw[dotted] (Q3)--(R3);
\draw[dotted] (Q4)--(R4);
    \end{tikzpicture}
 
The dotted correspondence shows that the field on the left is the fixed field of the subgroup on the right. 

\medskip 
\noindent {\bf $\F_{p^6}$:} The intermediary fields in this case are $\F_{p^2}$ and $\F_{p^3}$. The Galois group is $\langle\sigma\rangle$, where $\sigma$ is the Frobenius automorphism that takes $x$ to $x^p$. The order of $\sigma$ is 6. So its intermediary subgroups are $\langle \sigma^2\rangle$ which has order $3$ and $\langle \sigma^3\rangle$, which has order $2$:

 \begin{tikzpicture}
    \node (Q1) at (0,0) {$\F_p$};
    \node (Q2) at (2,2) {$\F_{p^3}$};
    \node (Q3) at (0,4) {$\F_{p^6}$};
    \node (Q4) at (-2,2) {$\F_{p^2}$};

    \draw (Q1)--(Q2);
    \draw (Q1)--(Q4);
    \draw (Q3)--(Q4);
    \draw (Q2)--(Q3);
    
    \node (R1) at (8,0) {$\langle\sigma\rangle$};
\node (R2) at (6,2) {$\langle\sigma^3\rangle$};
\node (R3) at (10,2) {$\langle \sigma^2\rangle$};
\node (R4) at (8,4) {$\{\mathfrak{1}\}$};

    \draw (R4)--(R2);
    \draw (R3)--(R4);
    \draw (R1)--(R3);
    \draw (R1)--(R2);

\draw[dotted] (Q1)--(R1);
\draw [blue,dotted] (Q4) to[out=20,in=70, distance=3cm ] (R2);
\draw [red,dotted] (Q2) to[out=-20,in=-70, distance=3cm ] (R3);
\draw[dotted] (Q3)--(R4);

        \end{tikzpicture}

\end{enumerate}
\end{example}


\section{Cyclotomics}
For $m\geq 1$, let $\omega_m \in \mathbb{C}$ be a primitive $m$th root of unity. We can take for example $\omega_m = e^{\frac{2\pi i}{m}}$. It is clear that $\omega_m$ is a generator for the multiplicative group of all $m$th roots of unity. 

From Number Theory, we know that if $\omega_m$ is a primitive root of unity, then $\omega_m^j$ is a primitive root of unity if and only if $GCD(j,m)=1$ or equivalently, $j\in \mathbb{Z}^*_m$, that is the multiplicative group of reduced residue classes modulo $m$. Thus, the set of all primitive $m$th roots of unit is given by 
$$\{\omega_m^j| j \in \mathbb{Z}^*_m\}. $$

We now take the polynomial 
$$\Phi_m(x) = \prod_{j\in \mathbb{Z}^*_m} (x-\omega_m^j),$$
which is a monic polynomial of degree $\phi(m)$, where $\phi$ is the Euler Totient function. This is called the $m$th {\it cyclotomic polynomial}. 

Some interesting facts about the cyclotomic polynomials are that 
\begin{itemize}
\item $\Phi_m(x) \in \mathbb{Z}[x]$.
\item $\Phi_m(x)|x^n-1$ in $\mathbb{Z}[x]$.
\item $\Phi_m(x)\nmid x^k-1$ in $\mathbb{Z}[x]$ for any $k<n$.
\item $\Phi_m(x)$ is irreducible over $\mathbb{Q}$ and hence over $\mathbb{Z}$ by Gauss' Lemma. 
\end{itemize}

\subsection{A list of some small cyclotomic polynomials}
We first observe that 
$\Phi_p(x) = \frac{x^p-1}{x-1}=x^{p-1}+x^{p-2}+\dots +x^2+x+1$ for any prime number, since $\phi(p)=p-1$ and so the primitive $p$th roots of unity cover all $p$th roots of unit except $1$. 
Another observation is that 
$$x^n-1 = \prod_{d|n}\Phi_d(x).$$
Based on this, we can list some of the cyclotomic polynomials:
\begin{align*}
    \Phi_1(x)&=x-1\\
    \Phi_2(x)& =x+1\\
    \Phi_3(x)&=x^2+x+1\\
    \Phi_4(x)&=x^2+1\\
    \Phi_5(x)&=x^4+x^3+x^2+x+1\\
    \Phi_6(x)&=x^2-x+1\\
    \Phi_8(x)&=x^4+1\\
    \Phi_9(x)& = x^6+x^3+1.
\end{align*}
There is a recursive algorithm that we can use to find the cyclotomic polynomial $\Phi_m(x)$. For example, for $\Phi_9(x)$, we can use the observation above to write
$$x^9-1=\Phi_9(x)\Phi_3(x)\Phi_1(x) = \Phi_9(x)(x^2+x+1)(x-1),$$
from which we conclude that 
$$\Phi_9(x)=\frac{x^9-1}{x^3-1} = x^6+x^3+1.$$

In general we can write 
$$\Phi_m(x) = \frac{x^m-1}{\prod_{d|m,\:d<m}\Phi_d(x)}.$$
So, after finding the initial terms, we can find the subsequent terms. 

As a last example of this idea, let us find $\Phi_{10}(x)$. So we have
$$\Phi_{10}(x) = \frac{x^{10}-1}{\Phi_1(x)\Phi_2(x)\Phi_5(x)}
=\frac{x^{10}-1}{(x-1)(x+1)(x^4+x^3+x^2+x+1)}$$
$$=\frac{x^{10}-1}{(x^5-1)(x+1)} = \frac{x^5+1}{x+1} = x^4-x^3+x^2-x+1.$$
Using \begin{verbatim}cyclotomic polynomial n \end{verbatim} 
command in Wolfram, one can find the cyclotomic polynomial of degree $\phi(n)$. 
\subsection{Cyclotomic Fields}
\begin{definition}
The field $\mathcal{K}_m = \Q(\omega_m)$ is called the $m$th cyclotomic number field. 
\end{definition}
Clearly we have $\mathcal{K}_m \simeq \mathbb{Q}[x]/(\Phi_m(x))$.

From standard Galois Theory, we know that all the automorphisms of $\mathcal{K}_m$ that fix $\Q$ are given by the maps $\omega_m \mapsto \omega_m^j$, where $GCD(j,m)=1$. Thus there are exactly $n=\phi(m)$ automorphisms of the field. 

Also, since the degree of $\Phi_m(x)$ is $n=\phi_m$, we have $\mathcal{K}_m$ as a $\Q$-vector space of dimension $m$, with a basis given by $\{1, \omega_m, \omega_m^2, \dots, \omega_m^{n-1}\}$. Thus every element in $\mathcal{K}_m$ can be uniquely expressed as $\sum_{i=0}^{n-1}a_i\omega_m^i$, where $a_i \in \Q$. This gives a natural (additive) embedding of elements of $\mathcal{K}_m$ into $\mathbb{C}^n$:

\begin{definition}(Embedding by coefficients)
Define $\sigma:\mathcal{K}_m \rightarrow \mathbb{C}^n$ by 
$$\sigma(\mathbf{a}) = \sigma \left (\sum_{i=0}^{n-1}a_i\omega_m^i\right ) = (a_0, a_1, \dots, a_{n-1}).$$
\end{definition}

There is a different way to embed elements of $\mathcal{K}_m$ into $\mathbb{C}^n$, which is called the {\it canonical embedding}:
\begin{definition} (Canonical Embedding)
In this case what we do is we view $\mathbf{a} = \sum_{i=0}^{n-1}a_i\omega_m^i$ as a polynomial evaluated at $\omega_m$, i.e., for $\mathbf{a} \in \mathcal{K}_m$, we take
$p_{\mathbf{a}}(x) =a_0+a_1x+\dots +a_{n-1}x^{n-1}$. Then the embedding $\tau:\mathcal{K}_m \rightarrow \mathbb{C}^n$ is given by 
$$\tau(\mathbf{a}) = \left \{p_{\mathbf{a}}(\omega_m^j)\right\}_{j\in \Z_m^*}.$$
\end{definition}

While both $\sigma$ and $\tau$ embed elements of $\mathcal{K}_m$ into $\mathbb{C}^n$, there is a fundamental difference between them. 
While $\sigma$ is only additive, $\tau$ on the other hand is both additive and multiplicative since the components are just polynomial evaluations. What this means is that $\tau$ is actually a ring homomorphism.

\subsection{Cylcotomics Splitting Modulo Primes}
One of the important properties of cylotomics is how they lead to the formation of slots in the Chinese Remainder Theorem, when we consider them modulo a prime number. We start with the following theorem and then we will se some particular examples:
\begin{theorem}
Let $p$ be a prime number such that $p\nmid m$. Then 
$$\Phi_m(x) = f_1(x)f_2(x)\dots f_{\ell}(x)$$
in $\Z_p[x]$, where $f_i(x)$ are irreducible and pairwise relatively prime polynomials in $\Z_p[x]$. Moreover, each $f_i(x)$ is of degree $d$, where $d=\frac{\phi(m)}{\ell}$ and $d$ is the order of $p$ mod $m$. That is $d$ is the smallest positive integer such that $p^d\equiv 1\pmod{m}$.  
\end{theorem}
An immediate corrolary is the following:
\begin{corollary}
 If $p$ is a primitive root modulo $m$, then the polynomial $\Phi_m(x)$ will also be irreducible over $\Z_p$, since in this case $d=\phi(m)$. 
\end{corollary}
The other extreme case is when $d=1$, which happens $p\equiv 1 \pmod{m}$:
\begin{corollary}
If $p \equiv 1 \pmod{m}$, then $\Phi_{m}(x)$ splits into distinct linear factors modulo $p$. In this case we will have $$\Phi_m(x) = \prod_{i\in \Z_m^*}(x-a^i),$$
where $a$ is a primitive $m$th root of unity modulo $p$. 
\end{corollary}

\begin{example}
Let us consider several examples of what is described in the theorem and the corollaries:
\begin{enumerate}
    \item Consider $\Phi_{15}(x) = x^8-x^7+x^5-x^4+x^3-x+1$.
    If $p=31$ for example, then we have $31\equiv 1\pmod{15}$ and so we expect $\Phi_{15}(x)$ to split into linear factors in $\Z_{31}[x]$. Indeed, modulo $31$, we have 
    $$\Phi_{15}(x) = (x+3)(x+11)(x+12)(x+13)(x+17)(x+21)(x+22)(x+24). $$
    Note that $3$ is a primitive root modulo $31$, which means $3^2=9$ is a primitive $15$th root of unity. Thus we must have
    $$\Phi_{15}(x) = (x-9)(x-9^2)(x-9^4)(x-9^7)(x-9^8)(x-9^{11})(x-9^{13})(x-9^{14})$$
    $$=(x-9)(x-19)(x-20)(x-10)(x-28)(x-14)(x-18)(x-7)$$
    $$\equiv (x+22)(x+12)(x+11)(x+21)(x+3)(x+17)(x+13)(x+24) \pmod{31}$$
    $$(x+3)(x+11)(x+12)(x+13)(x+17)(x+21)(x+22)(x+24).$$
    
    Similarly, if we take $p=61$, we again have the same situation and this time, modulo $61$ we get the following factorization for $\Phi_{15}(x)$:
    $$\Phi_{15}(x)=(x+4)(x+5)(x+19)(x+36)(x+39)(x+45)(x+46)(x+49).$$
Now let us take $p=2$. Since $2^4$ is the first power of $2$ that is $1$ mod $15$, we see that the order of $2$ modulo $15$ is $4$ and hence we expect $\Phi_{15}(x)$ to split into two distinct irreducible factors in $\Z_2[x]$. Indeed we have 
$$\Phi_{15}(x) = (x^4+x+1)(x^4+x^3+1)$$ in $\Z_2[x]$. 

On the other hand, since $11^2=121\equiv 1 \pmod{15}$, the order of $11$ mod $15$ is $2$, so we expect $\Phi_{15}(x)$ to split into $4$ irreducible polynomials of degree $2$. Indeed 
$$\Phi_{15}(x) = (x^2+3x+9)(x^2+4x+5)(x^2+5x+3)(x^2+9x+4)$$
in $\Z_{11}[x]$. 

\item Let us consider $\Phi_{17}(x)$ this time. Since $17$ is prime, we know 
$$\Phi_{17}(x) = x^{16}+x^{15}+x^{14}+x^{13}+x^{12}+x^{11}+x^{10}+x^{9}+x^8+x^7+x^6+x^5+X^4+x^3+x^2+x+1.$$

If we choose $p=103$ which is 1 modulo $17$, then $\Phi_{17}(x)$ should split into linear factors and indeed, modulo $103$, we have
$$\Phi_{17}(x) = (x+3)(x+10)(x+22)(x+24)(x+27)(x+31)(x+37)(x+39)(x+42)$$
$$(x+69)(x+73)(x+80)(x+89)(x+90)(x+94)(x+95).$$

If we choose $p=2$, then since $2^4\equiv-1\pmod{17}$, we have order of $2$ modulo $17$ is $8$ and so $\Phi_{17}(x)$ should split into two distinct irreducible factors of degree $8$ over $\Z_2$. Indeed we have 
$$\Phi_{17}(x) = (x^8+x^5+x^4+x^3+1)(x^8+x^7+x^6+x^4+x^2+x+1)$$
in $\Z_2[x]$. 

If we choose $p=3$, then we see that $3$ is a primitive element modulo $17$ and so $\Phi_{17}(x)$ is irreducible in $\Z_{3}[x]$. 

If we choose $p=13$, then we see that order of $13$ modulo $17$ is $4$ and hence we expect $\Phi_{17}(x)$ to split into $4$ irreducible factors of degree $4$. Indeed we have 
$$\Phi_{17}(x) = (x^4+2x^3+5x^2+2x+1)(x^4+6x^3+6x^2+6x+1)(x^4+9x^3+9x+1)(x^4+10x^3+9x^2+10x+1)$$ in $\Z_{13}[x]$. 
\end{enumerate}
\end{example}

Recall that if $\Phi_m(x) = f_1(x)f_2(x)\dots f_{\ell}(x)$, in $\Z_p[x]$, where $deg(f_i)=d$, then by Chinese Remainder Theorem we have 
$$\Z_p[x]/(\Phi_m(x)) \simeq \Z_p[x]/(f_1(x))\times \Z_p[x]/(f_2(x)) \times \dots \times \Z_p[x]/(f_{\ell}(x))$$ so that we have $\ell$ slots of width $d$ each to work with in this case. Hence cyclotomics together with prime modulus directly determine the slot structure and the slot widths. Moreover, since each $f_i(x)$ is irreducible in $\Z_p[x]$, we have each ideal $(f_i(x))$ as a maximal ideal, which implies each factor in the in the CRT decomposition is a finite field of the same size and hence we have 
$$\Z_p[x]/(f_i(x)) \simeq \F_{p^d}, \:\:\:\: i=1, 2, \dots, \ell.$$

Of particular interest is the case in use when $\Phi_{2d}(x)=x^d+1$, with $d$ being a power of $2$, which is what is used in \cite{GHS} to set up the plaintext and ciphertext spaces, namely modulo reduction of the polynomial rings $\Z[x]/(\Phi_{2d}(x))$. So, as we saw above, if $p$ is a prime such that $p\equiv 1 \pmod{2d}$, then $x^d+1$ splits into $d$ linear factors, which means, 
$$\Z_p[x]/(x^d+1) \simeq \Z_p[x]/(x-a)\times \Z_p[x]/(x-a^3)\times \dots \times\Z_p[x]/(x-a^{2d-1}),$$
where $a$ is a primitive $2d$th root of unity modulo $p$. Note that $\Z_p[x]/(x-a^{2i-1})\simeq \Z_p$ for each $i=1,2, \dots, d$. Thus we get $d$ isomorphic copies of $\Z_p$ as the plaintext slots. 

\section{Automorphisms and Slot Permutations}
\subsection{The Main Set up}
Let $\mathcal{A} = \Z[x]/(\Phi_m(x))$. For simplicity we will be considering $\mathcal{A}_p$, where $p$ is a prime number that does not divide $m$. Note that $\Phi_m(x)$ is the $m$th cyclotomic polynomial, which is of degree $\phi(m)$, and it can be described as 
$$\Phi_m(x) = \prod_{j\in\Z_m^*}(x-\eta^j),$$
where $\eta$ is a primitive $m$th root of unity. 

If $d$ is the order of $p$ mod $m$, then we know $\Phi_m(x)$ splits into irreducible polynomials of degree $d$ over $\Z_p$:
$$\Phi_m(x) = F_1(x)F_2(x)\dots F_n(x),$$
where $n=\phi(m)/d$ and each $F_i(x)$ is a distinct irreducible polynomial over $\Z_p$. 

By CRT we have a $\Z_p$-algebra isomorphism (that is a ring isomorphism that leaves elements of $\Z_p$ fixed)
$$\mathcal{A}_p \rightarrow \Z_p[x]/(F_1(x))\times \Z_p[x]/(F_2(x)) \times \dots \times \Z_p[x]/(F_n(x))$$ given by 
$$f(x)\mapsto \left (f(x) \pmod{F_1(x)}, f(x) \pmod{F_2(x)}, \dots, f(x) \pmod{F_n(x)}\right).$$
Let us call $E = \Z_p[x]/(F_1(x))$. Note that $E$ is a finite field of order $p^d$ and since all finite fields of the same size are isomorphic it does not matter which factor we choose for $E$. So, in a sense we have an isomorphism of $\mathcal{A}_p$ to $E^n$. 
Letting $\eta = x \pmod{F_1(x)}$, we get $E=\Z_p[\eta]$. So $\eta$ can be viewed as a root of $F_1(x)$ in $E$. It is clear then that $\eta$ is a primitive $m$th root of unity so it can be taken to be the same $\eta$ that was used to build the cyclotomic in a sense.

Since the order of $p$ mod $m$ is $d$, we get $H=\langle \bar{p}\rangle$ as a cyclic subgroup of $\Z_m^*$ of order $d$. Namely, $H=\{1, \bar{p}, \bar{p}^2, \dots, \bar{p}^{d-1}\}$. 
So $\Z_m^*$ will be partitioned into $n$ different cosets of $H$. We denote the coset representatives by $k_1, k_2, \dots, k_n$ ($k_1=1$). Note that these can be ordered in such a way that $F_i(x)$ has $d$ roots, given by $\eta^k : k\in k_iH$ in $E$. 
This induces a $\Z_p$-algebra isomorphism between 
$\Z_p[x]/(F_i(x))$ and $E$ given by $f(x) \mapsto f(\eta^{k_i})$
Combining these, we get a $\Z_p$-algebra isomorphism:
$$\mathcal{A}_p\rightarrow E^n,$$ given by 
$$f(x) \mapsto \left (f(\eta^{k_1}), f(\eta^{k_2}), \dots, f(\eta^{k_n})\right ).$$

\subsection{Galois Automorpshims}
For $j\in \Z_m^*$, $\theta_j:\A_p\rightarrow \A_p$ given by 
$$\theta_j(f(x)) = f(x^j)$$ is a $\Z_p$-algebra isomorphism called ``Galois automorphism". Note that when $j\in H$, we get the Frobenius auotmorphisms. So Frobenius automorphisms are a special case of Galois auotmorphisms.
Applying this to the correspondence:
$$\A_p \longleftrightarrow \left (f(\eta^{k_1}), f(\eta^{k_2}), \dots, f(\eta^{k_n})\right ) \in E^n,$$ we get
$$\theta_j(f(x)) \in \A_p \longleftrightarrow \left (f(\eta^{jk_1}), f(\eta^{jk_2}), \dots, f(\eta^{jk_n})\right )\in E^n.$$
The idea is to choose $k_i$ carefully so that these correspond to some permutations of the slots. 

\subsection{The different cases}
There are several cases to consider when it comes to the coset representatives, but we will try to understand just the first case as it clearly shows how the permutations are done:
\\
{\bf Case 1:} The coset representatives are $1, g, g^2, \dots, g^{n-1}$ for some $g \in \Z_m^*$. First of all, in this case we must have $g^n\in H$. Why? Because $g^n\in \Z_m^*$, it must be in a coset, so we can assume $g^n\in g^iH$ for $0\leq i\leq n-1$. If $i=0$, then no problem. So we can assume $1\leq i\leq n-1$. Then we would have $g^{n-i} \in H$, which would contradict that $1, g, g^2, \dots, g^{n-1}$ all represent different cosets. What this means is that $g^n$ is given by $\bar{p}^s$ for some $s$. 

So now we can write the $\Z_p$-algebra isomorphism as 
$$\A_p \longleftrightarrow \left (f(\eta), f(\eta^g), \dots, f(\eta^{g^{n-2}}), f(\eta^{g^{n-1}})\right ) \in E^n.$$
So if we apply $\theta_g$ to $f$, we get 
$$\theta_g(f(x))\in \A_p \longleftrightarrow \left (f(\eta^g), f(\eta^{g^2}), \dots, f(\eta^{g^{n-1}}), f(\eta^{g^{n}})\right ) \in E^n. $$

So now if, as Shai and Victor describe it, this is a good dimension, then we have $g^n=1$, in which case we get $$ \left (f(\eta^g), f(\eta^{g^2}), \dots, f(\eta^{g^{n-1}}), f(\eta)\right ),$$
which gives us a true rotation of the slots. 
The following picture shows how this happens. 

\bigskip

\begin{tikzpicture}
\tikzstyle{point2}=[ball color=green, circle, draw=black, inner sep=0.2cm]
 
    \node[fill=cyan, text=black, rectangle,draw,  minimum width = 1.5cm, 
    minimum height = 1.5cm] (r1) at (0,0) {$f(\eta)$};

 \node[fill=red, text=black, rectangle,draw,  minimum width = 1.5cm, 
    minimum height = 1.5cm] (r2) at (3,0) {$f(\eta^g)$};

\filldraw[black] (4.5,0) circle (2pt); 
 
\filldraw[black] (6,0) circle (2pt); 

\filldraw[black] (7.5,0) circle (2pt);
 \node[fill=magenta, text=black, rectangle,draw,  minimum width = 1.5cm, 
    minimum height = 1.5cm] (rk) at (9,0) {$f(\eta^{g^{n-2}})$};

\node[fill=yellow, text=black, rectangle,draw,  minimum width = 1.5cm, 
    minimum height = 1.5cm] (rk1) at (12,0) {$f(\eta^{g^{n-1}})$};

\node (X) at (-3,0) [point2] {$\theta_g$};
\draw[-stealth, line width=1mm] (X)--(r1);

    \node[fill=red, text=black, rectangle,draw,  minimum width = 1.5cm, 
    minimum height = 1.5cm] (q1) at (0,-4) {$f(\eta^g)$};

 \node[fill=orange, text=black, rectangle,draw,  minimum width = 1.5cm, 
    minimum height = 1.5cm] (q2) at (3,-4) {$f(\eta^{g^2})$};

\filldraw[black] (4.5,-4) circle (2pt); 
 
\filldraw[black] (6,-4) circle (2pt); 

\filldraw[black] (7.5,-4) circle (2pt);
 \node[fill=yellow, text=black, rectangle,draw,  minimum width = 1.5cm, 
    minimum height = 1.5cm] (qk) at (9,-4) {$f(\eta^{g^{n-1}})$};

 \node[fill=cyan, text=black, rectangle,draw,  minimum width = 1.5cm, 
    minimum height = 1.5cm] (qk1) at (12,-4) {$f(\eta)$};

\draw[-stealth, line width=1mm] (r1)--(q1);
\draw[-stealth, line width=1mm] (r2)--(q2);
\draw[-stealth, line width=1mm] (rk)--(qk);
\draw[-stealth, line width=1mm] (rk1)--(qk1);
\end{tikzpicture}

\bigskip


If $g^n\neq 1$, then we still have $g^n\in H$, which implies $g^n=p^s$ for some $s$. In that case we get $f(\eta^{g^n}) = \theta_p^s(f(\eta))$, so it is a Frobenius automorphism applied to $f(\eta)$. In a sense we don't get a true rotation but rather a ``perturbed" rotation. However, there is still hope. If $f(\eta)$ happens to be in $\Z_p$, then since the Frobenius automorphisms leave elements of $\Z_p$ fixed we would have $\theta_p^s(f(\eta))=f(\eta)$ and so we again get a true rotation. 

If that does not work, we can still get around it by a form of ``masking". For example for this case we can take 
$$\mathbf{M}_1 \in \A_p \longleftrightarrow (1,1,\dots, 1,0) \in E^n.$$
Then $$1-\mathbf{M}_1 \in \A_p \longleftrightarrow (0,0,\dots, 0,1) \in E^n.$$

So, if we have 
$$\mathbf{a}\in \A_p \longleftrightarrow (\alpha_0,\alpha_1,\dots, \alpha_{n-1}) \in E^n,$$
then 
$$\mathbf{M}_1\cdot \theta_g(\mathbf{a}) \in \A_p \longleftrightarrow (\alpha_1,\alpha_2,\dots, \alpha_{n-1},0) \in E^n,$$
and 
$$(1-\mathbf{M}_1)\cdot \theta_{g^{1-n}}(\mathbf{a}) \in \A_p \longleftrightarrow (0,0,\dots, 0,\alpha_0) \in E^n.$$
Thus,  $\mathbf{M}_1\cdot \theta_g(\mathbf{a})+(1-\mathbf{M}_1)\cdot \theta_{g^{1-n}}(\mathbf{a})$ gives us an element of $\A_p$ whose slots are rotated one position. 

\subsection{An example} Let us look at a concrete example.
Let $m=16$ so that $\Phi_m(x)=x^8+1$ and $p=3$. Since the order of $3$ mod 8 is $4$, $d=4$ and $n=2$.Factorizing $\Phi_{16}(x)$ in $\Z_3[x]$, we get 
$$x^8+1 = (x^4+x^2+2)(x^4+2x^2+2).$$
$H=\langle 3\rangle = \{1,3,9,11\}$ and letting $g=5$, we get 
$\Z_{16}^* = H\cup 5H$. Note that $5^2=9 \in H$ but $5^2 \neq 1$ in $\Z_{16}^*$.
So we get a situation that is similar to what we just saw above. 
Letting $\eta$ be a root of $x^4+x^2+2$ in $E = \Z_3[x]/(x^4+x^2+2)$, we see that 
$$f(x) \in \A_p \longleftrightarrow \left (f(\eta),f(\eta^5)\right ) \in E^2.$$ Note that $\eta$ is a primitive $16$th root of $1$. 
In this case applying $\sigma_5$ we get 
$$\theta_5(f(x)) \in \A_p \longleftrightarrow \left (f(\eta^5),f(\eta^{25})\right ) = (f(\eta^5), f(\eta^9)) =(f(\eta^5), \theta_3^2(f(\eta)))\in E^2.$$
So we see that the last slot has the Frobenius automorphism applied to it. 

Just to see what the Frobenius automorphism does to the function,let us apply $\theta_3$:
$$\theta_3(f(x)) \in \A_p \longleftrightarrow \left (f(\eta^3),f(\eta^{15})\right ) \in E^2,$$
which has no connection to the rotations of the original element. 

\subsection{To Summarize}
For rotations, the coset decomposition of $\Z_m^{*}$ into cosets of $H=\langle p\rangle$ is essential. The Galois automorphisms labeled at coset representatives provide with the permutations rather than the Frobenius automorphisms. The Frobenius automorphisms seem to cause intra slot changes rather than permutations between slots. 

\begin{thebibliography}{99}

\bibitem{GHS} C. Gentry, S. Halevi and N. Smart, ``Fully Homomorphic 
Encryption with Polylog Overhead"

\bibitem{Des} S. Halevi and V. Shoup, ``HELib Design Principles"
\end{thebibliography}

\end{document} 